\section{Basic tactics}\label{sec:book}



A sizable fraction of proof scripts consists of steps that do not
"prove" anything new, but instead perform menial bookkeeping tasks
such as selecting the names of constants and assumptions or splitting
conjuncts. Although they are logically trivial, bookkeeping steps are
extremely important because they define the structure of the data-flow
of a proof script. This is especially true for reflection-based
proofs, which often involve large numbers of constants and
assumptions.  Good bookkeeping consists in always explicitly declaring
(i.e., naming) all new constants and assumptions in the script, and
systematically pruning irrelevant constants and assumptions in the
context. This is essential in the context of an interactive
development environment (IDE), because it facilitates navigating the
proof, allowing to instantly "jump back" to the point at which a
questionable assumption was added, and to find relevant assumptions by
browsing the pruned context.  While novice or casual \Coq{} users may
find the automatic name selection feature convenient, the usage of
such a feature severely undermines the readability and maintainability
of proof scripts, much like automatic variable declaration in programming
languages. The \ssr{} tactics are therefore designed to support
precise bookkeeping and to eliminate name generation heuristics.
The bookkeeping features of \ssr{} are implemented as tacticals (or
pseudo-tacticals), shared across most \ssr{} tactics, and thus form
the foundation of the \ssr{} proof language.

\subsection{Bookkeeping}\label{ssec:profstack}

During the course of a proof \Coq{} always present the user with
a \emph{sequent} whose general form is
\begin{displaymath}\begin{array}{l}
\arrayrulecolor{dkviolet}
c_i\ \C{:}\ T_i \\
\dots\\
d_j\ \C{:=}\ e_j\ \C{:}\ T_j \\
\dots\\
F_k\ \C{:}\ P_k \\
\dots \\[3pt]
\hline\hline\\[-8pt]
\C{forall}\ \C{(}x_\ell\ \C{:}\ T_\ell\C{)}\ \dots,\\
\C{let}\ y_m\ \C{:=}\ b_m\ \C{in}\ \dots\ \C{in}\\
P_n\ \C{->}\ \dots\ \C{->}\ C
\end{array}\end{displaymath}
The \emph{goal} to be proved appears below the double line; above the line is
the \emph{context} of the sequent, a set of declarations of
\emph{constants}~$c_i$, \emph{defined constants}~$d_i$, and
\emph{facts}~$F_k$ that can be used to prove the goal (usually, $T_i,
T_j\;:\;\C{Type}$ and $P_k\;:\;\C{Prop}$). The various kinds of
declarations can come in any order. The top part of the context
consists of declarations produced by the \C{Section} commands
\C{Variable}, \C{Let}, and \C{Hypothesis}. This \emph{section context}
is never affected by the \ssr{} tactics: they only operate on the
the lower part --- the \emph{proof context}.
As in the figure above, the goal often decomposes into a series of
(universally) quantified \emph{variables}
$\C{(}x_\ell\;\C{:}\;T_\ell\C{)}$, local \emph{definitions}
$\C{let}\;y_m\:\C{:=}\;b_m\;\C{in}$, and \emph{assumptions}
$P_n\;\C{->}$, and a \emph{conclusion}~$C$ (as in the context, variables,
definitions, and assumptions can appear in any order). The conclusion
is what actually needs to be proved --- the rest of the goal can be
seen as a part of the proof context that happens to be ``below the line''.

However, although they are logically equivalent, there are fundamental
differences between constants and facts on the one hand, and variables
and assumptions on the others. Constants and facts are
\emph{unordered}, but \emph{named} explicitly in the proof text;
variables and assumptions are \emph{ordered}, but \emph{unnamed}: the
display names of variables may change at any time because of
$\alpha$-conversion.

Similarly, basic deductive steps such as \C{apply} can only operate on
the goal because the Gallina terms that control their action (e.g.,
the type of the lemma used by \C{apply}) only provide unnamed bound
variables.\footnote{Thus scripts that depend on bound variable names, e.g.,
via \C{intros} or \C{with}, are inherently fragile.}  Since the proof
script can only refer directly to the context, it must constantly
shift declarations from the goal to the context and conversely in
between deductive steps.

In \ssr{} these moves are performed by two \emph{tacticals} `\C{=>}'
and `\C{:}', so that the bookkeeping required by a deductive step can
be directly associated to that step, and that tactics in an \ssr{}
script correspond to actual logical steps in the proof rather than
merely shuffle facts. Still, some isolated bookkeeping is unavoidable,
such as naming variables and assumptions at the beginning of a proof.
\ssr{} provides a specific \C{move} tactic for this purpose.
%; about one out of every six tactics is a \C{move}.

Now \C{move} does essentially nothing: it is mostly a placeholder for
`\C{=>}' and `\C{:}'. The `\C{=>}' tactical moves variables, local
definitions, and assumptions to the context, while the `\C{:}'
tactical moves facts and constants to the goal. For example, the proof
of\footnote{The name \C{subnK} reads as
``right cancellation rule for \C{nat} subtraction''.}
\begin{lstlisting}
  Lemma |*subnK*| : forall m n, n <= m -> m - n + n = m.
\end{lstlisting}
might start with
\begin{lstlisting}
  move=> m n le_n_m.
\end{lstlisting}
where \C{move} does nothing, but \L|=> m n le_m_n| changes the
variables and assumption of the goal in the constants \C{m n : nat}
and the fact \L|le_n_m : n <= m|, thus exposing the conclusion\\
 \C{m - n + n = m}. % This is exactly what the specialized \Coq{} tactic
% \L|intros m n le_m_n| would do, but `\C{=>}' is much more general
% (see~\ref{ssec:intro}).
\todo{Shall we mention \C{intro}?}

The `\C{:}' tactical is the converse of `\C{=>}': it removes facts
and constants from the context by turning them into variables and assumptions.
Thus
\begin{lstlisting}
  move: m le_n_m.
\end{lstlisting}
turns back \C{m} and \L|le_m_n| into a variable and an assumption, removing
them from the proof context, and changing the goal to
\begin{lstlisting}
  forall m, n <= m -> m - n + n = m.
\end{lstlisting}
which can be proved by induction on \C{n} using \C{elim n}.
% The specialized \Coq{} tactic \C{revert} does exactly this,
% but `\C{:}' is much more general (see~\ref{ssec:discharge}).
\todo{Shall we mention revert?}

\noindent
Because they are tacticals, `\C{:}' and `\C{=>}' can be combined, as in
\begin{lstlisting}
  move: m le_n_m => p le_n_p.
\end{lstlisting}
simultaneously renames \L|m| and \L|le_m_n| into \L|p| and \L|le_p_n|,
respectively, by first turning them into unnamed variables, then
turning these variables back into constants and facts.

Furthermore, \ssr{} redefines the basic \Coq{} tactics \C{case},
\C{elim}, and \C{apply} so that they can take better advantage of
'\C{:}' and `\C{=>}'. In there \ssr{} variants, these tactic operate
on the first  variable or constant of the goal and they do not use or
change the proof context. The `\C{:}' tactical is used to operate on
an element in the context. For instance the proof of \C{subnK} could
continue with
\begin{lstlisting}
  elim: n.
\end{lstlisting}
instead of \C{elim n}; this has the advantage of
removing \C{n} from the context. Better yet, this \C{elim} can be combined
with previous \C{move} and with the branching version of the \C{=>} tactical
(described in~\ref{ssec:intro}),
to encapsulate the inductive step in a single command:
\begin{lstlisting}
  elim: n m le_n_m => [|n IHn] m => [_ | lt_n_m].
\end{lstlisting}
which breaks down the proof into two subgoals,
\begin{lstlisting}
  m - 0 + 0 = m
\end{lstlisting}
given \C{m : nat}, and
\begin{lstlisting}
  m - S n + S n = m
\end{lstlisting}
given \C{m n : nat}, \L|lt_n_m : S n <= m|, and
\begin{lstlisting}
  IHn : forall m, n <= m -> m - n + n = m.
\end{lstlisting}
The '\C{:}' and `\C{=>}' tacticals can be explained very simply
if one views the goal as a stack of variables and assumptions piled
on a conclusion:
\begin{itemize}
\item \L|/*tactic*/: $a$ $b$ $c$| pushes the context constants $a$, $b$, $c$
as goal variables \emph{before} performing \L|/*tactic*/|.
\item \L|/*tactic*/=> $a$ $b$ $c$| pops the top three goal variables as
context constants $a$, $b$, $c$, \emph{after} \L|/*tactic*/|
has been performed.
\end{itemize}
These pushes and pops do not need to balance out as in the examples above,
so
\begin{lstlisting}
  move: m le_n_m => p.
\end{lstlisting}
would rename \C{m} into \C{p}, but leave an extra assumption \C{n <= p}
in the goal.

Basic tactics like \C{apply} and \C{elim} can also be used without the
'\C{:}' tactical: for example we can directly start a proof of \C{subnK}
by induction on the top variable \C{m} with
\begin{lstlisting}
  elim=> [|m IHm] n le_n.
\end{lstlisting}

\noindent
The general form of the localization tactical \C{in} is also best
explained in terms of the goal stack:
\begin{lstlisting}
  /*tactic*/ in a H1 H2 *.
\end{lstlisting}
is basically equivalent to
\begin{lstlisting}
  move: a H1 H2; /*tactic*/ => a H1 H2.
\end{lstlisting}
with two differences: the \C{in} tactical will preserve the body of \C{a} if
\C{a} is a defined constant, and if the `\C{*}' is omitted it
will use a temporary abbreviation to hide the statement of the goal
from \C{/*tactic*/}.

The general form of the \C{in} tactical can be used directly with
the \C{move}, \C{case} and \C{elim} tactics, so that one can write
\begin{lstlisting}
  elim: n => [|n IHn] in m le_n_m *.
\end{lstlisting}
instead of
\begin{lstlisting}
  elim: n m le_n_m => [|n IHn] m le_n_m.
\end{lstlisting}
This is quite useful for inductive proofs that involve many facts.

\noindent See section \ref{ssec:gloc} for the general syntax and presentation
of the \L+in+ tactical.

\subsection{The defective tactics}\label{ssec:basictac}

In this section we briefly present the three basic tactics performing
context manipulations and the main backward chaining tool.

\subsubsection*{The \C{move} tactic.}\label{sssec:move}
The \C{move} tactic, in its
defective form, behaves like the primitive \C{hnf} \Coq{} tactic. For
example, such a defective:
\begin{lstlisting}
  move.
\end{lstlisting}
exposes the first assumption in the goal, i.e. its changes the goal
\L+~ False+ into \L+False -> False+.

More precisely, the \C{move} tactic inspects the goal and does nothing
(\C{idtac}) if an introduction step is possible, i.e. if the
goal is a product or a \L+let $\dots$ in+, and performs \C{hnf}
otherwise.

Of course this tactic is most often used in combination with the
bookkeeping tacticals (see section \ref{ssec:intro} and
\ref{ssec:discharge}). These combinations mostly subsume the \C{intros},
\C{generalize}, \C{rename}, \C{clear} and
\textcolor{dkblue}{\texttt{pattern}} tactics.


\subsubsection*{The \C{case} tactic.} The \C{case} tactic performs
\emph{primitive case analysis} on (co)inductive types; specifically,
it destructs the top variable or assumption of the goal,
exposing its constructor(s) and its arguments, as well as setting the value
of its type family indices if it belongs to a type family
(see section \ref{ssec:typefam}).

% The \C{case} tactic could often be replaced by
% the \C{move} tactic combined with destructing \emph{i-items} (see
% \ref{ssec:intro}), but this should only be done when the destructed
% (co)inductive type has only one
% constructor and no family indices.

The \ssr{} \C{case} tactic has a special behavior on
equalities.
If the top assumption of the goal is an equality, the \C{case} tactic
``destructs'' it as a set of equalities between the constructor
arguments of its left and right hand sides, as per the
tactic \C{injection}.
For example, \C{case} changes the goal
\begin{lstlisting}
  (x, y) = (1, 2) -> G.
\end{lstlisting}
into
\begin{lstlisting}
  x = 1 -> y = 2 -> G.
\end{lstlisting}

Note also that the case of \ssr{} performs \C{False}
elimination, even if no branch is generated by this case operation.
Hence the command:
\begin{lstlisting}
  case.
\end{lstlisting}
on a goal of the form \C{False -> G} will succeed and prove the goal.

\subsubsection*{The \C{elim} tactic.} The \C{elim} tactic performs
inductive elimination on inductive types.
The defective:
\begin{lstlisting}
  elim.
\end{lstlisting}
tactic performs inductive elimination on a goal whose top assumption
has an inductive type. For example on goal of the form:
\begin{lstlisting}
  forall n : nat, m <= n
\end{lstlisting}
 in a context containing {m : nat}, the
\begin{lstlisting}
  elim.
\end{lstlisting}
tactic produces two goals,
\begin{lstlisting}
  m <=  0
\end{lstlisting}
on one hand and
\begin{lstlisting}
  forall n : nat, m <= n -> m <= S n
\end{lstlisting}
on the other hand.


\subsubsection*{The \C{apply} tactic.}\label{sssec:apply}
The \C{apply} tactic is the main
backward chaining tactic of the proof system. It takes as argument any
\C{/*term*/} and applies it to the goal.
Assumptions in the type of \C{/*term*/} that don't directly match the
goal may generate one or more subgoals.

In fact the \ssr{} tactic:
\begin{lstlisting}
  apply.
\end{lstlisting}
is a synonym for:
\begin{lstlisting}
  intro top; first [refine top | refine (top _) |  refine (top _ _) | $\dots$]; clear top.
\end{lstlisting}
where \C{top} is fresh name, and the sequence of \C{refine} tactics
tries to catch the appropriate number of wildcards to be inserted.
Note that this use of the \C{refine} tactic implies that the tactic
tries to match the goal up to expansion of
constants and evaluation of subterms.

\ssr{}'s \C{apply} has a special behaviour on goals containing
existential metavariables of sort \C{Prop}.  Consider the
following example:
\begin{lstlisting}
Goal (forall y, 1 < y -> y < 2 -> exists x : 'I_3, x > 0).
move=> y y_gt1 y_lt2; apply: (ex_intro _ (@Ordinal _ y _)).
  by apply: leq_trans y_lt2 _.
by move=> y_lt3; apply: leq_trans _ y_gt1.
\end{lstlisting}
Note that the last \C{_} of the tactic \C{apply: (ex_intro _ (@Ordinal _ y _))}
represents a proof that \C{y < 3}. Instead of generating the following
goal
\begin{lstlisting}
   0 < Ordinal (n:=3) (m:=y) ?54
\end{lstlisting}
\noindent the system tries to prove \C{y < 3} calling the \C{trivial}
tactic. If it succeeds, let's say because the context contains
\C{H : y < 3}, then the system generates the following goal:
\begin{lstlisting}
   0 < Ordinal (n:=3) (m:=y) H
\end{lstlisting}
\noindent Otherwise the missing proof is considered to be irrelevant, and
is thus discharged generating the following goals:
\begin{lstlisting}
   y < 3
   forall Hyp0 : y < 3, 0 < Ordinal (n:=3) (m:=y) Hyp0
\end{lstlisting}
Last, the user can replace the \C{trivial} tactic by defining
an \Ltac{} expression named \C{ssrautoprop}.


\subsection{Discharge}\label{ssec:discharge}
The general syntax of the discharging tactical `\C{:}' is:
\begin{displaymath}
\N{tactic}\ [\N{ident}]
            \C{:}\ \N{d-item}_1\ \dots \ \N{d-item}_n\ [\N{clear-switch}]
\end{displaymath}
where $n > 0$, and $\N{d-item}$ and $\N{clear-switch}$ are defined as
\begin{eqnarray*}
\N{d-item}& \equiv& [\N{occ-switch} \;|\; \N{clear-switch}]\;\N{term}\\
\N{clear-switch}& \equiv& \C{\{} \N{ident}_1\, \ldots\,  \N{ident}_m \C{\}}
\end{eqnarray*}
with the following requirements:
\begin{itemize}
\item $\N{tactic}$ must be one of the four basic tactics described
      in~\ref{ssec:basictac}, i.e., \C{move}, \C{case}, \C{elim} or \C{apply},
      the \C{exact} tactic (section \ref{ssec:termin}),
      the \C{congr} tactic (section \ref{ssec:congr}), or the application
      of the \emph{view} tactical `\C{/}' (section \ref{ssec:assumpinterp})
      to one of \C{move}, \C{case}, or \C{elim}.
\item The optional $\N{ident}$ specifies \emph{equation generation}
      (section \ref{ssec:equations}), and is only allowed if $\N{tactic}$
      is \C{move}, \C{case} or \C{elim}, or the application of the
      view tactical `\C{/}' (section \ref{ssec:assumpinterp})
      to \C{case} or \C{elim}.
\item An $\N{occ-switch}$ selects occurrences of $\N{term}$,
      as in \ref{sssec:occselect}; $\N{occ-switch}$ is not allowed if
      $\N{tactic}$ is \C{apply} or \C{exact}.
\item A clear item $\N{clear-switch}$ specifies facts and constants to be
  deleted from the proof context (as per the \C{clear} tactic).
\end{itemize}
The `\C{:}' tactical first \emph{discharges} all the $\N{d-item}$s,
right to left, and then performs $\N{tactic}$, i.e., for each $\N{d-item}$,
starting with $\N{d-item}_n$:
\begin{enumerate}
\item The \ssr{} matching algorithm described in section~\ref{ssec:set}
      is used to find occurrences of $\N{term}$ in the goal,
      after filling any holes `\C{_}' in $\N{term}$; however if $\N{tactic}$
      is \C{apply} or \C{exact} a different matching algorithm,
      described below, is used
      \footnote{Also, a slightly different variant may be used for the first
      $\N{d-item}$ of \C{case} and \C{elim}; see section~\ref{ssec:typefam}.}.
\item~\label{enum:gen} These occurrences are replaced by a new
  variable; in particular,
      if $\N{term}$ is a fact, this adds an assumption to the goal.
\item~\label{enum:clr} If $\N{term}$ is \emph{exactly} the name of a constant
      or fact in the proof context, it is deleted from the context,
      unless there is an $\N{occ-switch}$.
\end{enumerate}
Finally, $\N{tactic}$ is performed just after $\N{d-item}_1$ has been
generalized ---
that is, between steps \ref{enum:gen} and \ref{enum:clr} for $\N{d-item}_1$.
The names listed in the final $\N{clear-switch}$ (if it is present)
are cleared first, before $\N{d-item}_n$ is discharged.

\noindent
Switches affect the discharging of a $\N{d-item}$ as follows:
\begin{itemize}
\item An $\N{occ-switch}$ restricts generalization (step~\ref{enum:gen})
      to a specific subset of the occurrences of $\N{term}$, as per
      \ref{sssec:occselect}, and prevents clearing (step~\ref{enum:clr}).
\item All the names specified by a $\N{clear-switch}$ are deleted from the
      context in step~\ref{enum:clr}, possibly in addition to $\N{term}$.
\end{itemize}
For example, the tactic:
\begin{lstlisting}
  move: n {2}n (refl_equal n).
\end{lstlisting}
\begin{itemize}
\item first generalizes \C{(refl_equal n : n = n)};
\item then generalizes the second occurrence of \C{n}.
\item finally generalizes all the other occurrences of \C{n},
      and clears \C{n} from the proof context
      (assuming \C{n} is a proof constant).
\end{itemize}
Therefore this tactic changes any goal \C{G} into
\begin{lstlisting}
 forall n n0 : nat, n = n0 -> G.
\end{lstlisting}
where the name \C{n0} is picked by the \Coq{} display function,
and assuming \C{n} appeared only in~\C{G}.

Finally, note that a discharge operation generalizes defined constants
as variables, and not as local definitions. To override this behavior,
prefix the name of the local definition with a \C{@},
like in \C{move: @n}.

This is in contrast with the behavior of the \C{in} tactical (see section
\ref{ssec:gloc}), which preserves local definitions by default.

\subsubsection*{Clear rules}

The clear step will fail if $\N{term}$ is a proof constant that
appears in other facts; in that case either the facts should be
cleared explicitly with a $\N{clear-switch}$, or the clear step should be
disabled. The latter can be done by adding an $\N{occ-switch}$ or simply by
putting parentheses around $\N{term}$: both
\begin{lstlisting}
  move: (n).
\end{lstlisting}
and
\begin{lstlisting}
  move: {+}n.
\end{lstlisting}
generalize \L+n+ without clearing \L+n+ from the proof context.

The clear step will also fail if the $\N{clear-switch}$ contains a
$\N{ident}$ that is not in the \emph{proof} context.
Note that \ssr{} never clears a section constant.

If $\N{tactic}$ is \L+move+ or \L+case+ and an equation $\N{ident}$ is given,
then clear (step~\ref{enum:clr}) for $\N{d-item}_1$ is suppressed
(see section \ref{ssec:equations}).

\subsubsection*{Matching for \C{apply} and \C{exact}}\label{sss:strongapply}

The matching algorithm for $\N{d-item}$s of the \ssr{} \L+apply+ and
\C{exact} tactics
exploits the type of $\N{d-item}_1$ to interpret
wildcards in the other $\N{d-item}$ and to determine which occurrences of
these should be generalized.
Therefore, $\N{occur switch}$es are not needed for \L+apply+ and \L+exact+.

Indeed, the \ssr{} tactic \L+apply: H x+ is equivalent to
\begin{lstlisting}
  refine (@H _ $\dots$ _ x); clear H x
\end{lstlisting}
with an appropriate number of wildcards between \L+H+ and~\L+x+.

Note that this means that matching for \L+apply+ and \L+exact+ has
much more context to interpret wildcards; in particular it can accommodate
the `\C{_}' $\N{d-item}$, which would always be rejected after `\L+move:+'.
For example, the tactic
\begin{lstlisting}
  apply: trans_equal (Hfg _) _.
\end{lstlisting}
transforms the goal \L+f a = g b+, whose context contains
\L+(Hfg : forall x, f x = g x)+, into \L+g a = g b+.
This tactic is equivalent (see section \ref{ssec:profstack}) to:
\begin{lstlisting}
  refine (trans_equal (Hfg _) _).
\end{lstlisting}
and this is a common idiom for applying transitivity on the left hand side
of an equation.

\subsubsection*{The \C{abstract} tactic}\label{ssec:abstract}
The \C{abstract} assigns an abstract constant previously introduced with
the \C{[: name ]} intro pattern (see section~\ref{ssec:intro},
page~\pageref{ssec:introabstract}).
In a goal like the following:
\begin{lstlisting}
 m : nat
 abs : <hidden>
 n : nat
 =============
 m < 5 + n
\end{lstlisting}
The tactic \C{abstract: abs n} first generalizes the goal with respect to
\C{n} (that is not visible to the abstract constant \C{abs}) and then
assigns \C{abs}.  The resulting goal is:
\begin{lstlisting}
 m : nat
 n : nat
 =============
 m < 5 + n
\end{lstlisting}
Once this subgoal is closed, all other goals having \C{abs} in their context
see the type assigned to \C{abs}.  In this case:
\begin{lstlisting}
 m : nat
 abs : forall n, m < 5 + n
\end{lstlisting}

For a more detailed example the user should refer to section~\ref{sssec:have},
page~\pageref{sec:havetransparent}.

\subsection{Introduction}\label{ssec:intro}

The application of a tactic to a given goal can generate
(quantified) variables, assumptions, or definitions, which the user may want to
\emph{introduce} as new facts, constants or defined constants, respectively.
If the tactic splits the goal into several subgoals,
each of them may require the introduction of different constants and facts.
Furthermore it is very common to immediately decompose
or rewrite with an assumption instead of adding it to the context,
as the goal can often be simplified and even
proved after this.

All these operations are performed by the introduction tactical
`\C{=>}', whose general syntax is
% liste indic�e cf explication. Mettre la version ^+ dans le synopsis
\begin{displaymath}
  \N{tactic}\ \C{=>}\ \N{i-item}_1 \dots \N{i-item}_n
\end{displaymath}
where $\N{tactic}$ can be any tactic, $n > 0$ and
\begin{eqnarray*}
  \N{i-item}& \equiv&
    \N{i-pattern} \;|\; \N{s-item} \;|\; \N{clear-switch} \;|\; \C{/}\N{term} \\
  \N{s-item}& \equiv& \C{/=} \;|\; \C{//} \;|\; \C{//=} \\
  \N{i-pattern}& \equiv&
           \N{ident} \;|\; \C{_} \;|\; \C{?} \;|\; \C{*}\;|\;
           [\N{occ-switch}]\C{->} \;|\; [\N{occ-switch}]\C{<-} \;|\; \\
   &&      \C{[} \N{i-item}^*_1 \;\C{|}\;\dots ;\C{|}\;\N{i-item}^*_m \C{]}
	   \;|\;\C{-}\;|\;\C{[:}\N{ident}^+\C{]}
\end{eqnarray*}
%subject to the condition that at least every other $\N{i-item}$ should
%be an $\N{i-pattern}$ (for any two consecutive $\N{i-item}_i\
%\N{i-item}_{i+1}$, either $\N{i-item}_i$ or $\N{i-item}_{i+1}$ is an
%$\N{i-pattern}$).

The `\C{=>}' tactical first executes $\N{tactic}$, then the
$\N{i-item}$s, left to right, i.e., starting from $\N{i-item}_1$.  An
$\N{s-item}$ specifies a simplification operation; a $\N{clear
switch}$ specifies context pruning as in~\ref{ssec:discharge}. The
$\N{i-pattern}$s can be seen as a variant of \emph{intro patterns}:
each performs an introduction operation, i.e., pops some variables or
assumptions from the goal.

An $\N{s-item}$ can simplify the set of subgoals or the subgoal themselves:
\begin{itemize}
\item \C{//} removes all the ``trivial'' subgoals that can be resolved by
      the \ssr{} tactic \C{done} described in~\ref{ssec:termin}, i.e., it
      executes \C{try done}.
\item \C{/=} simplifies the goal by performing partial evaluation, as
   per the tactic \C{simpl}.\footnote{Except \C{/=} does not
   expand the local definitions created by the \ssr{} \C{in} tactical.}
\item \L+//=+ combines both kinds of simplification; it is equivalent
    to \L+/= //+, i.e., \L+simpl; try done+.
\end{itemize}
When an $\N{s-item}$ bears a $\N{clear-switch}$, then the $\N{clear-switch}$ is
executed \emph{after} the $\N{s-item}$, e.g., \L|{IHn}//| will solve
some subgoals, possibly using the fact \L|IHn|, and will erase \L|IHn|
from the context of the remaining subgoals.

The last entry in the $\N{i-item}$ grammar rule, \C{/}$\N{term}$,
represents a view (see section~\ref{sec:views}). If $\N{i-item}_{k+1}$
is a view $\N{i-item}$, the view is applied to the assumption in top
position once $\N{i-item}_1 \dots \N{i-item}_k$ have been performed.

The view is applied to the top assumption.

\ssr{} supports the following $\N{i-pattern}$s:
\begin{itemize}
\item $\N{ident}$ pops the top variable, assumption, or local definition into
      a new constant, fact, or defined constant $\N{ident}$, respectively.
      Note that defined constants cannot be introduced when
      $\delta$-expansion is required to expose the top variable or assumption.
\item \C{?} pops the top variable into an anonymous constant or fact,
      whose name is picked by the tactic interpreter.
      \ssr{} only generates names that
      cannot appear later in the user script.\footnote{\ssr{} reserves
      all identifiers of the form ``\C{_x_}'', which is used for such
      generated names.}
\item \C{_} pops the top variable into an anonymous constant that will be
     deleted from
     the proof context of all the subgoals produced by the \C{=>} tactical.
     They should thus never be displayed, except in an error message
     if the constant is still actually used in the goal or context after
     the last $\N{i-item}$ has been executed ($\N{s-item}$s can erase goals
     or terms where the constant appears).
\item \C{*} pops all the remaining apparent variables/assumptions
     as anonymous constants/facts. Unlike \C{?} and \C{move} the \C{*}
     $\N{i-item}$ does not expand definitions in the goal to expose
     quantifiers, so it may be useful to repeat a \C{move=> *} tactic,
     e.g., on the goal
\begin{lstlisting}
  forall a b : bool, a <> b
\end{lstlisting}
a first \C{move=> *} adds only \C{_a_ : bool} and \C{_b_ : bool} to
the context; it takes a second \C{move=> *} to add
\C{_Hyp_ : _a_ = _b_}.
\item $[\N{occ-switch}]$\C{->} (resp. $[\N{occ-switch}]$\C{<-})
  pops the top assumption
  (which should be a rewritable proposition) into an anonymous fact,
  rewrites (resp. rewrites right to left) the goal with this fact
  (using the \ssr{} \C{rewrite} tactic described in section~\ref{sec:rw},
  and honoring the optional occurrence selector),
  and finally deletes the anonymous fact from the context.
\item\C{[ $\N{i-item}^*_1$ | $\dots$ | $\N{i-item}^*_m$ ]},
   when it is the very \emph{first} $\N{i-pattern}$ after $\N{tactic}\;\C{=>}$
   tactical \emph{and} $\N{tactic}$ is not a \C{move}, is a \emph{branching}
   $\N{i-pattern}$. It executes
   the sequence $\N{i-item}^*_i$ on the $i^{\rm th}$
   subgoal produced by $\N{tactic}$. The execution of $\N{tactic}$
   should thus generate exactly $m$
   subgoals, unless the \C{[$\dots$]} $\N{i-pattern}$ comes after an initial
   \C{//} or \C{//=} $\N{s-item}$ that closes some of the goals produced by
   $\N{tactic}$, in which case exactly $m$ subgoals should remain after the
   $\N{s-item}$, or we have the trivial branching $\N{i-pattern}$ \C{[]},
   which always does nothing, regardless of the number of remaining subgoals.
\item\C{[ $\N{i-item}^*_1$ | $\dots$ | $\N{i-item}^*_m$ ]}, when it is
   \emph{not} the first $\N{i-pattern}$ or when $\N{tactic}$ is a
   \C{move}, is a \emph{destructing} $\N{i-pattern}$.  It starts by
   destructing the top variable, using the \ssr{} \C{case} tactic
   described in~\ref{ssec:basictac}.  It then behaves as the
   corresponding branching $\N{i-pattern}$, executing the sequence
   $\N{i-item}^*_i$ in the $i^{\rm th}$ subgoal generated by the case
   analysis; unless we have the trivial destructing $\N{i-pattern}$
   \C{[]}, the latter should generate exactly $m$ subgoals, i.e., the
   top variable should have an inductive type with exactly $m$
   constructors.\footnote{More precisely, it should have a quantified
   inductive type with $a$ assumptions and $m - a$ constructors.}
   While it is good style to use the $\N{i-item}^*_i$
   to pop the variables and assumptions corresponding to each constructor,
   this is not enforced by \ssr{}.
\item\C{-} does nothing, but counts as an intro pattern. It can also
  be used to force the interpretation of
   \C{[ $\N{i-item}^*_1$ | $\dots$ | $\N{i-item}^*_m$ ]}
   as a case analysis like in \C{move=> -[H1 H2]}. It can also be used
   to indicate explicitly the link between a view and a like in
   \C{move=> /eqP-H1}. Last, it can serve as a separator between
   views. Section~\ref{ssec:multiview} explains in which respect
   the tactic \C{move=> /v1/v2} differs from the tactic
   \C{move=> /v1-/v2}.
\item\C{[: $\N{ident}^+$ ]} introduces in the context an abstract constant
   for each $\N{ident}$.  Its type has to be fixed later on by using
   the \C{abstract} tactic (see page~\pageref{ssec:abstract}).  Before then
   the type displayed is \C{<hidden>}.\label{ssec:introabstract}
\end{itemize}
Note that \ssr{} does not support the syntax
$\C{(}\N{ipat}\C{,}\dots\C{,}\N{ipat}\C{)}$ for destructing
intro-patterns.

Clears are deferred until the end of the intro pattern. For
example, given the goal:
\begin{lstlisting}
x, y : nat
==================
0 < x = true -> (0 < x) && (y < 2) = true
\end{lstlisting}
the tactic \C{move=> \{x\} ->} successfully rewrites the goal and
deletes \C{x} and the anonymous equation. The goal is thus turned into:
\begin{lstlisting}
y : nat
==================
true && (y < 2) = true
\end{lstlisting}
If the cleared names are reused in the same intro pattern, a renaming
is performed behind the scenes.

Facts mentioned in a clear switch must be valid
names in the proof context (excluding the section context).

The rules for interpreting branching and destructing $\N{i-pattern}$
are motivated by the fact that it would be pointless to have a branching
pattern if $\N{tactic}$ is a \C{move}, and in most of the remaining cases
$\N{tactic}$ is \C{case} or \C{elim}, which implies destruction.
The rules above imply that
\begin{lstlisting}
  move=> [a b].
  case=> [a b].
  case=> a b.
\end{lstlisting}
are all equivalent, so which one to use is a matter of style;
\C{move} should be used for casual decomposition,
such as splitting a pair, and \C{case} should be used for actual decompositions,
in particular for type families (see~\ref{ssec:typefam})
and proof by contradiction.

The trivial branching $\N{i-pattern}$ can be used to force the branching
interpretation, e.g.,
\begin{lstlisting}
  case=> [] [a b] c.
  move=> [[a b] c].
  case;  case=> a b c.
\end{lstlisting}
are all equivalent.

\subsection{Generation of equations}\label{ssec:equations}

The generation of named equations option stores the definition of a
new constant as an equation. The tactic:
\begin{lstlisting}
  move En: (size l) => n.
\end{lstlisting}
where \C{l} is a list, replaces \C{size l} by \C{n} in the goal and
adds the fact \C{En : size l = n} to the context.
 This is quite different from:
\begin{lstlisting}
  pose n := (size l).
\end{lstlisting}
which generates a definition \C{n := (size l)}. It is not possible to
generalize or
rewrite such a definition; on the other hand, it is automatically
expanded during
computation, whereas expanding the equation \C{En} requires explicit
rewriting.

The use of this equation name generation option with a \C{case} or an
\C{elim} tactic changes the status of the first \iitem{}, in order to
deal with the possible parameters of the constants introduced.

On the
goal \C{a <> b} where \C{a, b} are natural numbers, the tactic:
\begin{lstlisting}
  case E : a => [|n].
\end{lstlisting}
generates two subgoals. The equation \L+E : a = 0+  (resp. \C{E : a =
  S n}, and the constant \C{n : nat}) has been added to
the context of the goal \L+0 <> b+ (resp. \C{S n  <> b}).

If the user does not provide a branching \iitem{} as first \iitem{},
or if the \iitem{} does not provide enough names for the arguments of
a constructor,
then the constants generated are introduced under fresh \ssr{} names.
For instance, on the goal \C{a <> b}, the tactic:
\begin{lstlisting}
  case E : a => H.
\end{lstlisting}
also generates two subgoals, both requiring a proof of \L+False+.
 The hypotheses \C{E : a = 0} and
\C{H : 0 = b} (resp. \C{E : a = S _n_} and
\C{H : S _n_ = b}) have been added to the context of the first
subgoal (resp. the second subgoal).

Combining the generation of named equations mechanism with the
\L+case+ tactic strengthens the power of a case analysis. On the other
hand, when combined with the \L+elim+ tactic, this feature is mostly
useful for
debug purposes, to trace the values of decomposed parameters and
pinpoint failing branches.

% This feature is also useful
% to analyse and debug generate-and-test style scripts that prove program
% properties by generating a large set of input patterns and uniformly
% solving the corresponding subgoals by computation and rewriting, e.g,

% \begin{lstlisting}
%   case: et => [|e' et]; first by case: s.
%   case: e => //; case: b; case: w.
% \end{lstlisting}
% If the above sequence fails, then it's easy enough to replace the line
% above with
% \begin{lstlisting}
%   case: et => [|e' et].
%   case Ds: s; case De: e => //; case Db: b; case Dw: w=> [|s' w'] //=.
% \end{lstlisting}
% Then the first subgoal that appears will be the failing one, and the
% equations \C{Ds}, \C{De}, \C{Db}
% and \C{Dw} will pinpoint its branch. When the constructors of
% the decomposed type have arguments (like \C{w : (seq nat)}
% above) these need to be
% introduced in order to generate the equation, so there should
% always be an explicit \iitem{} (\C{\[|s' w'\]} above) that
% assigns names to these arguments. If this \iitem{}
% is omitted the arguments are introduced with default
% name; this
% feature should be
% avoided except for quick debugging runs (it has some uses in complex tactic
% sequences, however).



\subsection{Type families}\label{ssec:typefam}

When the top assumption of a goal has an inductive type, two
specific operations are possible: the case analysis performed by the
\L+case+ tactic, and the application of an induction principle,
performed by the \L+elim+ tactic. When this top assumption has an
inductive type, which is moreover an instance of a type family, \Coq{}
may need help from the user to specify which occurrences of the parameters
of the type should be substituted.

A specific \C{/} switch indicates the type family parameters of the
type of a \ditem{} immediately following this \C{/} switch, using the
syntax:

$$[\C{case}|\C{elim}]: \N{d-item}^+ / \N{d-item}^*$$

The $\N{d-item}$s on the right side of the \C{/} switch are discharged
as described in section \ref{ssec:discharge}. The case analysis or
elimination will be done on the type of the top assumption after these
discharge operations.

Every $\N{d-item}$ preceding the \C{/} is interpreted as arguments of this
type, which should be an instance of an inductive type family. These terms are
not actually generalized, but rather selected for substitution. Occurrence
switches can be used to restrict the substitution.  If a $\N{term}$ is left
completely implicit (e.g. writing just $\C{\_}$), then a pattern is inferred
looking at the type of the top assumption.  This allows for the compact syntax
\C{case: \{2\}\_ / eqP}, were \C{\_} is interpreted as \C{(\_ == \_)}. Moreover
if the $\N{d-item}$s list is too short, it is padded with an initial
sequence of $\C{\_}$ of the right length.

Here is a small example on lists. We define first a function which
adds an element at the end of a given list.
\begin{lstlisting}
  Require Import List.

  Section LastCases.
  Variable A : Type.

  Fixpoint |*add_last*|(a : A)(l : list A): list A :=
  match l with
    |nil => a :: nil
    |hd :: tl => hd :: (add_last a tl)
  end.
\end{lstlisting}
Then we define an inductive predicate for
case analysis on lists according to their last element:
\begin{lstlisting}
  Inductive |*last_spec*| : list A -> Type :=
    | LastSeq0 : last_spec nil
    | LastAdd s x : last_spec (add_last x s).

  Theorem |*lastP*| : forall l : list A, last_spec l.
\end{lstlisting}
Applied to the goal:
\begin{lstlisting}
  Goal forall l : list A, (length l) * 2 = length (app l l).
\end{lstlisting}
the command:
\begin{lstlisting}
  move=> l; case: (lastP l).
\end{lstlisting}
generates two subgoals:
\begin{lstlisting}
  length nil * 2 = length (nil ++ nil)
\end{lstlisting}
and
\begin{lstlisting}
  forall (s : list A) (x : A),
    length (add_last x s) * 2 = length (add_last x s ++ add_last x s)
\end{lstlisting}
both having \L+l : list A+ in their context.

Applied to the same goal, the command:
\begin{lstlisting}
  move=> l; case: l / (lastP l).
\end{lstlisting}
generates the same subgoals but \L+l+ has been cleared from both
contexts.

Again applied to the same goal, the command:
\begin{lstlisting}
  move=> l; case: {1 3}l / (lastP l).
\end{lstlisting}
generates the subgoals \L-length l * 2 = length (nil ++ l)- and
\L-forall (s : list A) (x : A), length l * 2 = length (add_last x s++l)-
where the selected occurrences on the left of the \C{/} switch have
been substituted with \L+l+ instead of being affected by the case
analysis.

The equation name generation feature combined with a type family \C{/}
  switch generates an equation for the \emph{first} dependent d-item
specified by the user.
Again starting with the above goal, the command:
\begin{lstlisting}
  move=> l; case E: {1 3}l / (lastP l)=>[|s x].
\end{lstlisting}
adds \C{E : l = nil} and \C{E : l = add_last x s},
respectively, to the context of the two subgoals it generates.

There must be at least one \emph{d-item} to the left of the \C{/}
switch; this prevents any
confusion with the view feature. However, the \ditem{}s to the right of
the \C{/} are optional, and if they are omitted the first assumption
provides the instance of the type family.

The equation always refers to the first \emph{d-item} in the actual
tactic call, before any padding with initial $\C{\_}$s. Thus, if an
inductive type has two family parameters, it is possible to have
\ssr{} generate an equation for the second one by omitting the pattern
for the first; note however that this will fail if the type of the
second parameter depends on the value of the first parameter.
