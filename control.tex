\section{Control flow}
\subsection{Indentation and bullets}\label{ssec:indent}

The linear development of \Coq{} scripts gives little information on
the structure of the proof. In addition, replaying a proof after some
changes in the statement to be proved will usually not display information to
distinguish between the various branches of case analysis for instance.

To help the user in this organization of the proof script at
development time, \ssr{} provides some bullets to highlight the
structure of branching proofs. The available bullets are \C{-},
\C{+} and \C{*}. Combined with tabulation, this lets us highlight four
nested levels of branching; the most we have ever
needed is three. Indeed, the use of ``simpl and closing'' switches, of
terminators (see above section \ref{ssec:termin}) and selectors (see
 section \ref{ssec:select}) is powerful enough
to avoid most of the time more than two levels of indentation.

Here is a fragment of such a structured script:

\begin{lstlisting}
case E1: (abezoutn _ _) => [[| k1] [| k2]].
- rewrite !muln0 !gexpn0 mulg1 => H1.
  move/eqP: (sym_equal F0); rewrite -H1 orderg1 eqn_mul1.
  by case/andP; move/eqP.
- rewrite muln0 gexpn0 mulg1 => H1.
  have F1: t %| t * S k2.+1 - 1.
    apply: (@dvdn_trans (orderg x)); first by rewrite F0; exact: dvdn_mull.
    rewrite orderg_dvd; apply/eqP; apply: (mulgI x).
    rewrite -{1}(gexpn1 x) mulg1 gexpn_add leq_add_sub //.
    by move: P1; case t.
  rewrite dvdn_subr in F1; last by exact: dvdn_mulr.
  + rewrite H1 F0 -{2}(muln1 (p ^ l)); congr (_ * _).
    by apply/eqP; rewrite -dvdn1.
  + by move: P1; case: (t) => [| [| s1]].
- rewrite muln0  gexpn0 mul1g => H1.
$\dots$
\end{lstlisting}

\subsection{Terminators}\label{ssec:termin}
To further structure scripts, \ssr{}
supplies \emph{terminating} tacticals to explicitly close off tactics.
When replaying scripts, we then have the nice property that
an error immediately occurs when a closed tactic fails to prove its
subgoal.

It is hence recommended practice that the proof of any subgoal should
end with a tactic which \emph{fails if it does not solve the current
  goal}. Standard \Coq{} already provides some tactics of this kind,
like \C{discriminate}, \C{contradiction} or \C{assumption}.

\ssr{} provides a generic tactical which turns any tactic into a closing
one. Its general syntax is:
\begin{lstlisting}
  by $\N{tactic}$.
\end{lstlisting}
The \Ltac{} expression:
\begin{lstlisting}
  by [$\N{tactic}_1$ | [$\N{tactic}_2$ | $\dots$].
\end{lstlisting}
is equivalent to:
\begin{lstlisting}
  [by $\N{tactic}_1$ | by $\N{tactic}_2$ | $\dots$].
\end{lstlisting}
and this form should be preferred to the former.

In the script provided as example in section \ref{ssec:indent}, the
paragraph corresponding to each sub-case ends with a tactic line prefixed
with a \C{by}, like in:
\begin{lstlisting}
  by apply/eqP; rewrite -dvdn1.
\end{lstlisting}

The \C{by} tactical is implemented using the user-defined,
and extensible \C{done} tactic. This \C{done} tactic tries to solve
the current goal by some trivial means and fails if it doesn't succeed.
Indeed, the tactic expression:
\begin{lstlisting}
  by $\N{tactic}$.
\end{lstlisting}
is equivalent to:
\begin{lstlisting}
  $\N{tactic}$; done.
\end{lstlisting}
Conversely, the tactic
\begin{lstlisting}
  by [ ].
\end{lstlisting}
is equivalent to:
\begin{lstlisting}
  done.
\end{lstlisting}
The default implementation of the \C{done} tactic, in the {\tt
  ssreflect.v} file, is:

\begin{lstlisting}
Ltac done :=
  trivial; hnf; intros; solve
   [ do ![solve [trivial | apply: sym_equal; trivial]
         | discriminate | contradiction | split]
   | case not_locked_false_eq_true; assumption
   | match goal with H : ~ _ |- _ => solve [case H; trivial] end ].
\end{lstlisting}

The lemma \C{|*not_locked_false_eq_true*|} is needed to discriminate
\emph{locked} boolean predicates (see section \ref{ssec:lock}).
The iterator tactical \C{do} is presented in section
\ref{ssec:iter}.
This tactic can be customized by the user, for instance to include an
\C{auto} tactic.

A natural and common way of closing a goal is to apply a lemma which
is the \C{exact} one needed for the goal to be solved. The defective
form of the tactic:
\begin{lstlisting}
  exact.
\end{lstlisting}
is equivalent to:
\begin{lstlisting}
  do [done | by move=> top; apply top].
\end{lstlisting}
where \C{top} is a fresh name affected to the top assumption of the goal.
This applied form is supported by the  \C{:} discharge tactical, and
the tactic:
\begin{lstlisting}
  exact: MyLemma.
\end{lstlisting}
is equivalent to:
\begin{lstlisting}
  by apply: MyLemma.
\end{lstlisting}
(see section \ref{sss:strongapply} for the documentation of the \C{apply:}
combination).

\textit{Warning} The list of tactics, possibly chained by
semi-columns, that follows a \C{by} keyword is considered as a
parenthesized block
applied to the current goal. Hence for example if the tactic:
\begin{lstlisting}
   by rewrite my_lemma1.
\end{lstlisting}
succeeds, then the tactic:
\begin{lstlisting}
   by rewrite my_lemma1; apply my_lemma2.
\end{lstlisting}
usually fails since it is equivalent to:
\begin{lstlisting}
   by (rewrite my_lemma1; apply my_lemma2).
\end{lstlisting}

\subsection{Selectors}\label{ssec:select}

When composing tactics, the two tacticals \C{first} and
\C{last} let the user restrict the application of a tactic to only one
of the subgoals generated by the previous tactic. This
covers the frequent cases where a tactic generates two subgoals one of
which can be easily disposed of.

This is an other powerful way of linearization of scripts, since it
happens very often that a trivial subgoal can be solved in a less than
one line tactic. For instance, the tactic:
\begin{lstlisting}
  $\N{tactic}_1$; last by $\N{tactic}_2$.
\end{lstlisting}
tries to solve the last subgoal generated by $\N{tactic}_1$ using the
$\N{tactic}_2$, and fails if it does not succeeds. Its analogous
\begin{lstlisting}
  $\N{tactic}_1$; first by $\N{tactic}_2$.
\end{lstlisting}
tries to solve the first subgoal generated by $\N{tactic}_1$ using the
tactic $\N{tactic}_2$, and fails if it does not succeeds.


\ssr{} also offers an extension of this facility, by supplying
tactics to \emph{permute}  the subgoals generated by a tactic.
The tactic:
\begin{lstlisting}
  $\N{tactic}$; last first.
\end{lstlisting}
inverts the order of the subgoals generated by $\N{tactic}$. It is
equivalent to:
\begin{lstlisting}
  $\N{tactic}$; first last.
\end{lstlisting}

More generally, the tactic:
\begin{lstlisting}
  $\N{tactic}$; last $\N{num}$ first.
\end{lstlisting}
where $\N{num}$ is
standard \Coq{} numeral or \Ltac{} variable denoting\\
a standard \Coq{} numeral having the value $k$,
rotates the $n$ subgoals $G_1,
\dots, G_n$ generated by $\N{tactic}$. The first subgoal becomes
$G_{n + 1 - k}$ and the circular order of subgoals remains unchanged.

Conversely, the tactic:
\begin{lstlisting}
  $\N{tactic}$; first $\N{num}$ last.
\end{lstlisting}
rotates the $n$ subgoals $G_1,
\dots, G_n$ generated by \C{tactic} in order that the first subgoal
becomes $G_{k}$.

Finally, the tactics \C{last} and \C{first} combine with the
branching syntax of \Ltac{}:
if the tactic $\N{tactic}_0$ generates $n$
subgoals on a given goal, then the tactic
\begin{lstlisting}
  $tactic_0$; last $\N{num}$ [$tactic_1$|$\dots$|$tactic_m$] || $tactic_{m+1}$.
\end{lstlisting}
where $\N{num}$ denotes the integer $k$ as above, applies $tactic_1$ to the
$n -k + 1$-th goal, $\dots tactic_m$ to the $n -k + 2 - m$-th
goal and \C{$tactic_{m+1}$} to the others.

For instance, the script:
\begin{lstlisting}
  Inductive test : nat ->  Prop :=
    C1 : forall n, test n | C2 : forall n, test n |
    C3 : forall n, test n | C4 : forall n, test n.

  Goal forall n, test n -> True.
  move=> n t; case: t; last 2 [move=> k| move=> l]; idtac.
\end{lstlisting}

creates a goal with four subgoals, the first and the last being
\C{nat -> True}, the second and the third being \C{True} with
respectively \C{k : nat} and \C{l : nat} in their context.

\subsection{Iteration}\label{ssec:iter}

\ssr{} offers an accurate control on the repetition of
tactics, thanks to the \C{do} tactical, whose general syntax is:
  $$\C{do}\ [\N{mult}]\C{[}\N{tactic}_1 \C{|} \dots \C{|} \N{tactic}_n\C{]}$$
where $\N{mult}$ is a \emph{multiplier}.

Brackets can only be omitted if a single tactic is given \emph{and} a
multiplier is present.

A tactic of the form:
\begin{lstlisting}
  do [$tactic_1$ | $\dots$ | $tactic_n$].
\end{lstlisting}
is equivalent to the standard \Ltac{} expression:
\begin{lstlisting}
  first [$tactic_1$ | $\dots$ | $tactic_n$].
\end{lstlisting}

The optional multiplier $\N{mult}$ specifies how many times
the action of $\N{tactic}$ should be repeated on the current subgoal.

There are four kinds of multipliers:
  \begin{itemize}
  \item \C{$n$!}: the step $\N{tactic}$ is repeated exactly $n$ times
    (where $n$ is a positive integer argument).
  \item \C{!}: the step $\N{tactic}$ is repeated as many times as possible,
    and done at least once.
  \item \C{?}: the step $\N{tactic}$ is repeated as many times as possible,
    optionally.
  \item \C{$n$?}: the step $\N{tactic}$ is repeated up to $n$ times,
    optionally.
  \end{itemize}

For instance, the tactic:
\begin{lstlisting}
  $\N{tactic}$; do 1?rewrite mult_comm.
\end{lstlisting}
rewrites at most one time the lemma \C{mult_com} in all the subgoals
generated by $\N{tactic}$, whereas the tactic:
\begin{lstlisting}
  $\N{tactic}$; do 2!rewrite mult_comm.
\end{lstlisting}
rewrites exactly two times the lemma \C{mult_com} in all the subgoals
generated by $\N{tactic}$, and fails if this rewrite is not possible
in some subgoal.

Note that the combination of multipliers and \C{rewrite} is so often
used that multipliers are in fact integrated to the syntax of the \ssr{}
\C{rewrite} tactic, see section \ref{sec:rw}.

\subsection{Localization}\label{ssec:gloc}

In sections \ref{ssec:loc} and \ref{ssec:profstack}, we have already
presented the \emph{localization} tactical \L+in+, whose general
syntax is:
$$\N{tactic}\ \C{in}\ \N{ident}^+ [\C{*}]$$
where $\N{ident}^+$ is a non empty list of fact
names in the context. On the left side of \C{in}, $\N{tactic}$ can be
\C{move}, \C{case}, \C{elim}, \C{rewrite}, \C{set},
 or any tactic formed with the general iteration tactical \C{do} (see
 section \ref{ssec:iter}).

The operation described by $\N{tactic}$ is performed in the facts
listed in $\N{ident}^+$ and in the goal if a \C{*} ends
the list.

The \C{in} tactical successively:
\begin{itemize}
\item generalizes the selected hypotheses, possibly ``protecting'' the
 goal if \C{*} is not present,
\item performs $\N{tactic}$, on the obtained goal,
\item reintroduces the generalized facts, under the same names.
\end{itemize}

This defective form of the \L+do+ tactical is useful to avoid clashes
between standard \Ltac{} \C{in} and the \ssr{} tactical \C{in}.
For example, in the following script:
\begin{lstlisting}
  Ltac |*mytac*| H := rewrite H.

  Goal forall x y, x = y -> y = 3 -> x + y  = 6.
  move=> x y H1 H2.
  do [mytac H2] in H1 *.
\end{lstlisting}
the last tactic rewrites the hypothesis \C{H2 : y = 3} both in
\C{H1 : x = y} and in the goal \C{x + y = 6}.

By default \C{in} keeps the body of local definitions. To erase
the body of a local definition during the generalization phase,
the name of the local definition must be written between parentheses,
like in \C{rewrite H in H1 (def_n) $\;\;$H2}.

From \ssr{} 1.5 the grammar for the \C{in} tactical has been extended
to the following one:

$$\N{tactic}\ \C{in}\ [ \N{clear-switch}\ |\
                        [\C{@}]\N{ident}\ |\
                        \C{(}\N{ident}\C{)}\ |\
                        \C{(}[\C{@}]\N{ident}\ \C{:=}\ \N{c-pattern}\C{)}
                      ]^+ [\C{*}]$$

In its simplest form the last option lets one rename hypotheses that can't be
cleared (like section variables).  For example \C{(y := x)} generalizes
over \C{x} and reintroduces the generalized
variable under the name \C{y} (and does not clear \C{x}).\\
For a more precise description the $\C{(}[\C{@}]\N{ident}\ \C{:=}\ \N{c-pattern}\C{)}$
item refer to the ``Advanced generalization'' paragraph at page~\pageref{par:advancedgen}.

\subsection{Structure}\label{ssec:struct}

Forward reasoning structures the script by explicitly specifying some
assumptions to be added to the proof context. It is closely associated
with the declarative style of proof, since an extensive use of these
highlighted statements
make the script closer to a (very detailed) text book proof.

Forward chaining tactics allow to state an intermediate lemma and start a
piece of script dedicated to the proof of this statement. The use of
closing tactics (see section \ref{ssec:termin}) and of
indentation makes syntactically explicit the portion of the script
building the proof of the intermediate statement.

\subsubsection*{The \C{have} tactic.}
\label{sssec:have}

The main \ssr{} forward reasoning tactic is the \C{have} tactic. It
can be use in two modes: one starts a new (sub)proof for an
intermediate result in the main proof, and the other
provides explicitly a proof term for this intermediate step.

In the first mode, the syntax of \C{have} in its defective form is:
\begin{lstlisting}
  have: $\N{term}$.
\end{lstlisting}
This tactic supports open syntax for $\N{term}$. Apart from the open
syntax, when $\N{term}$ does
not contain any wildcard, this tactic
is almost\footnote{The \C{assert} tactic creates a $\zeta$ redex,
  whereas the \C{have} tactic creates a $\beta$ redex, and it
  introduces the lemma under an automatically chosen fresh name.}
equivalent to the standard \Coq{}:
\begin{lstlisting}
  assert $\N{term}$.
\end{lstlisting}
Applied to a goal \C{G}, it generates a first subgoal requiring a
proof of $\N{term}$ in the context of \C{G}. The difference with
the standard \Coq{} tactic is that the second generated subgoal is of
the form \C{$\N{term}$ -> G}, where $\N{term}$ becomes the new top
assumption, instead of being introduced with a fresh name.

Like in the case of the \C{pose} tactic (see section \ref{ssec:pose}),
the types of the holes are abstracted in $\N{term}$.
For instance, the tactic:
\begin{lstlisting}
  have: _ * 0 = 0.
\end{lstlisting}
is equivalent to:
\begin{lstlisting}
  have: forall n : nat, n * 0 = 0.
\end{lstlisting}
The \C{have} tactic also enjoys the same abstraction mechanism as the
\C{pose} tactic for the non-inferred implicit arguments. For instance,
the tactic:
\begin{lstlisting}
  have: forall x y, (x, y) = (x, y + 0).
\end{lstlisting}
opens a new subgoal to prove that:

\noindent\C{forall (T : Type) (x : T) (y : nat), (x, y) = (x, y + 0)}


The behavior of the defective \C{have} tactic makes it possible to
generalize it in the
following general construction:
  $$\C{have }\ \;\N{i-item}^* \; [\N{i-pattern}] \;
     [\N{s-item}\;|\;\N{binder}^+]\;[\C{:} \N{term}_1]\;
     [\C{:=} \N{term}_2 \;|\; \C{by }\N{tactic}]$$

Open syntax is supported for $\N{term}_1$ and $\N{term}_2$. For the
description of
\iitem{}s and clear switches see section \ref{ssec:intro}.
The first mode of the \L+have+ tactic, which opens a sub-proof for an
intermediate result, uses tactics of the form:
\begin{lstlisting}
  have $\N{clear-switch}$ $\N{i-item}$ : /*term*/ by /*tactic*/.
\end{lstlisting}
which behave like:
\begin{lstlisting}
  have: /*term*/; first by /*tactic*/.
  move=> $\N{clear-switch}$ $\N{i-item}$.
\end{lstlisting}
Note that the $\N{clear-switch}$ \emph{precedes} the
$\N{i-item}$, which allows to reuse a name of the context, possibly used
by the proof of the assumption, to introduce the new assumption
itself.

Hence the standard \Coq{}:
\begin{lstlisting}
  assert $\N{term}$.
\end{lstlisting}
is in fact equivalent\footnote{again, except that the kind of redex
  created is different} up to the open syntax to:
\begin{lstlisting}
  have ?: $\N{term}$.
\end{lstlisting}

The \C{by} feature is especially convenient when the proof script of the
statement is very short, basically when it fits in one line like in:
\begin{lstlisting}
  have H : forall x y, x + y = y + x by move=> x y; rewrite addnC.
\end{lstlisting}

The possibility of using \iitem{}s supplies a very concise
syntax for the further use of the intermediate step. For instance,
\begin{lstlisting}
  have -> : forall x, x * a = a.
\end{lstlisting}
on a goal \C{G}, opens a new subgoal asking for a proof of
\C{forall x, x * a = a}, and a second subgoal in which the lemma
 \C{forall x, x * a = a} has been rewritten in the goal \C{G}. Note
 that in this last subgoal, the intermediate result does not appear in
 the context.
Note that, thanks to the deferred execution of clears, the following
idiom is supported (assuming \C{x} occurs in the goal only):
\begin{lstlisting}
  have {x} -> : x = y
\end{lstlisting}

An other frequent use of the intro patterns combined with \C{have} is the
destruction of existential assumptions like in the tactic:
\begin{lstlisting}
  have [x Px]: exists x : nat, x > 0.
\end{lstlisting}
which opens a new subgoal asking for a proof of \C{exists x : nat, x >
  0} and  a second subgoal in which the witness is introduced under
the name \C{x : nat}, and its property under the name \C{Px : x > 0}.

An alternative use of the \C{have} tactic is to provide the explicit proof
term for the intermediate lemma, using tactics of the form:
\begin{lstlisting}
  have $[\N{ident}]$ := $\N{term}$.
\end{lstlisting}
This tactic creates a new assumption of type the type of
$\N{term}$. If the
optional $\N{ident}$ is present, this assumption is introduced under
the name $\N{ident}$. Note that the body of the constant is lost for
the user.

Again, non inferred implicit arguments and explicit holes are abstracted. For
instance, the tactic:
\begin{lstlisting}
  have H := forall x, (x, x) = (x, x).
\end{lstlisting}
adds to the context \C{H : Type -> Prop}. This is a schematic example but
the feature is specially useful when the proof term to give involves
for instance a lemma with some hidden implicit arguments.

After the $\N{i-pattern}$, a list of binders is allowed.
For example, if \C{Pos_to_P} is a lemma that proves that
\C{P} holds for any positive, the following command:
\begin{lstlisting}
  have H x (y : nat) : 2 * x + y = x + x + y by auto.
\end{lstlisting}
will put in the context \C{H : forall x, 2 * x = x + x}. A proof term
provided after \C{:=} can mention these bound variables (that are
automatically introduced with the given names).
Since the $\N{i-pattern}$ can be omitted, to avoid ambiguity, bound variables
can be surrounded with parentheses even if no type is specified:
\begin{lstlisting}
  have (x) : 2 * x = x + x by auto.
\end{lstlisting}

The $\N{i-item}$s and $\N{s-item}$ can be used to interpret the
asserted hypothesis with views (see section~\ref{sec:views}) or
simplify the resulting goals.

The \C{have} tactic also supports a \C{suff} modifier which allows for
asserting that a given statement implies the current goal without
copying the goal itself. For example, given a goal \C{G} the tactic
\C{have suff H : P} results in the following two goals:
\begin{lstlisting}
   |- P -> G
   H : P -> G |- G
\end{lstlisting}
Note that \C{H} is introduced in the second goal. The \C{suff}
modifier is not compatible with the presence of a list of binders.

\subsubsection*{Generating \C{let in} context entries with \C{have}}
\label{sec:havetransparent}

Since \ssr{} 1.5 the \C{have} tactic supports a ``transparent'' modifier to
generate \C{let in} context entries: the \C{@} symbol in front of the context
entry name.  For example:

\begin{lstlisting}
have @i : 'I_n by apply: (Sub m); auto.
\end{lstlisting}
generates the following two context entry:
\begin{lstlisting}
i := Sub m proof_produced_by_auto : 'I_n
\end{lstlisting}

Note that the sub-term produced by \C{auto} is in general huge and
uninteresting, and hence one may want to hide it.

For this purpose the \C{[: name ]} intro pattern and the tactic
\C{abstract} (see page~\pageref{ssec:abstract}) are provided.
Example:
\begin{lstlisting}
have [:blurb] @i : 'I_n by apply: (Sub m); abstract: blurb; auto.
\end{lstlisting}
generates the following two context entries:
\begin{lstlisting}
blurb : (m < n) (*1*)
i := Sub m blurb : 'I_n
\end{lstlisting}
The type of \C{blurb} can be cleaned up by its annotations by just simplifying
it.  The annotations are there for technical reasons only.

When intro patterns for abstract constants are used in conjunction
with \C{have} and an explicit term, they must be used as follows:

\begin{lstlisting}
have [:blurb] @i : 'I_n := Sub m blurb.
  by auto.
\end{lstlisting}

In this case the abstract constant \C{blurb} is assigned by using it
in the term that follows \C{:=} and its corresponding goal is left to
be solved.  Goals corresponding to intro patterns for abstract constants
are opened in the order in which the abstract constants are declared (not
in the ``order'' in which they are used in the term).

Note that abstract constants do respect scopes.  Hence, if a variable
is declared after their introduction, it has to be properly generalized (i.e.
explicitly passed to the abstract constant when one makes use of it).
For example any of the following two lines:
\begin{lstlisting}
have [:blurb] @i k : 'I_(n+k) by apply: (Sub m); abstract: blurb k; auto.
have [:blurb] @i k : 'I_(n+k) := apply: Sub m (blurb k); first by auto.
\end{lstlisting}
generates the following context:
\begin{lstlisting}
blurb : (forall k, m < n+k) (*1*)
i := fun k => Sub m (blurb k) : forall k, 'I_(n+k)
\end{lstlisting}

Last, notice that the use of intro patterns for abstract constants is
orthogonal to the transparent flag \C{@} for \C{have}.

\subsubsection*{The \C{have} tactic and type classes resolution}
\label{ssec:havetcresolution}

Since \ssr{} 1.5 the \C{have} tactic behaves as follows with respect to type
classes inference.

\begin{itemize}
\item \C{have foo : ty.}
	Full inference for \C{ty}.
	The first subgoal demands a proof of such instantiated statement.
\item \C{have foo : ty := .}
	No inference for \C{ty}. Unresolved instances are quantified in
	\C{ty}.  The first subgoal demands a proof of such quantified
	statement.  Note that no proof term follows \C{:=}, hence two
	subgoals are generated.
\item \C{have foo : ty := t.}
	No inference for \C{ty} and \C{t}.
\item \C{have foo := t.}
	No inference for \C{t}. Unresolved instances are quantified in the
	(inferred) type of \C{t} and abstracted in \C{t}.
\end{itemize}

The behavior of \ssr{} 1.4 and below (never resolve type classes)
can be restored with the option \C{Set SsrHave NoTCResolution}.

\subsubsection*{Variants: the \C{suff} and  \C{wlog} tactics.}

As it is often the case  in mathematical textbooks, forward
reasoning may be used in slightly different variants.
One of these variants is to show that the intermediate step $L$
easily implies the initial goal $G$. By easily we mean here that
the proof of $L \Rightarrow G$ is shorter than the one of $L$
itself. This kind of reasoning step usually starts with:
``It suffices to show that \dots''.

This is such a frequent way of reasoning that \ssr{} has a variant of the
\C{have} tactic called \C{suffices} (whose abridged name is
\C{suff}). The \C{have} and \C{suff} tactics are equivalent and
have the same syntax but:
\begin{itemize}
\item the order of the generated subgoals is inversed
\item but the optional clear item is still performed in the
  \emph{second} branch. This means that the tactic:
\begin{lstlisting}
  suff {H} H : forall x : nat, x >= 0.
\end{lstlisting}
fails if the context of the current goal indeed contains an
assumption named \C{H}.
\end{itemize}
The rationale of this clearing policy is to make possible ``trivial''
refinements of an assumption, without changing its name in the main
branch of the reasoning.

The \C{have} modifier can follow the \C{suff} tactic.
For example, given a goal \C{G} the tactic
\C{suff have H : P} results in the following two goals:
\begin{lstlisting}
  H : P |- G
  |- (P -> G) -> G
\end{lstlisting}
Note that, in contrast with \C{have suff}, the name \C{H} has been introduced
in the first goal.

Another useful construct is reduction,
showing that a particular case is in fact general enough to prove
a general property. This kind of reasoning step usually starts with:
``Without loss of generality, we can suppose that \dots''.
Formally, this corresponds to the proof of a goal \C{G} by introducing
a cut \C{/*wlog_statement*/ -> G}. Hence the user shall provide a
proof for both \C{(/*wlog_statement*/ -> G) -> G} and
\C{/*wlog_statement*/ -> G}.

\ssr{} implements this kind of reasoning step through the \C{without loss}
tactic, whose short name is \C{wlog}.
The general syntax of \C{without loss} is:
\begin{lstlisting}
wlog $[$suff$][\N{clear-switch}][\N{i-item}]$ : $[\N{ident}_1 \dots\N{ident}_n]$ / $\N{term}$
\end{lstlisting}
where $\N{ident}_1\ \dots\ \N{ident}_n$ are identifiers for constants
in the context of the goal. Open syntax is supported for $\N{term}$.

In its defective form:
\begin{lstlisting}
  wlog: / $\N{term}$.
\end{lstlisting}
on a goal \C{G}, it creates two subgoals: a first one to prove the formula
\C{($\N{term}$ -> G) -> G} and a second one to prove the formula
\C{$\N{term}$ -> G}.

If the optional list $\N{ident}_1\ \dots\ \N{ident}_n$ is present on the left
side of \C{/}, these constants are generalized in the premise
\C{($\N{term}$ -> G)} of the first subgoal. By default the body of
local definitions  is erased. This behavior can be inhibited
prefixing the name of the local definition with the \C{@} character.

In the second subgoal, the tactic:
\begin{lstlisting}
  move=> $\N{clear-switch}$ $\N{i-item}$.
\end{lstlisting}
is performed if at least one of these optional switches is present in
the \C{wlog} tactic.

The \C{wlog} tactic is specially useful when a symmetry argument
simplifies a proof. Here is an example showing the beginning of the
proof that quotient and reminder of natural number euclidean division
are unique.
\begin{lstlisting}
  Lemma quo_rem_unicity: forall d q1 q2 r1 r2,
     q1*d + r1 = q2*d + r2 -> r1 < d -> r2 < d -> (q1, r1) =  (q2, r2).
  move=> d q1 q2 r1 r2.
  wlog: q1 q2 r1 r2 / q1 <= q2.
    by case (le_gt_dec q1 q2)=> H; last symmetry; eauto with arith.
\end{lstlisting}

The \C{wlog suff} variant is simpler, since it cuts
\C{/*wlog_statement*/} instead of \C{/*wlog_statement*/ -> G}. It thus
opens the goals \C{/*wlog_statement*/ -> G} and \C{/*wlog_statement*/}.

In its simplest form
the \C{generally have :$\dots$} tactic
is equivalent to \C{wlog suff :$\dots$} followed by \C{last first}.
When the \C{have} tactic
is used with the \C{generally} (or \C{gen}) modifier it accepts an
extra identifier followed by a comma before the usual intro pattern.
The identifier will name the new hypothesis in its more general form,
while the intro pattern will be used to process its instance.  For example:
\begin{lstlisting}
  Lemma simple n (ngt0 : 0 < n ) : P n.
  gen have ltnV, /andP[nge0 neq0] : n ngt0 / (0 <= n) && (n != 0).
\end{lstlisting}
The first subgoal will be
\begin{lstlisting}
  n : nat
  ngt0 : 0 < n
  ====================
  (0 <= n) && (n != 0)
\end{lstlisting}
while the second one will be
\begin{lstlisting}
  n : nat
  ltnV : forall n, 0 < n -> (0 <= n) && (n != 0)
  nge0 : 0 <= n
  neqn0 : n != 0
  ====================
  P n
\end{lstlisting}

\paragraph{Advanced generalization}\label{par:advancedgen}
The complete syntax for the items on the left hand side of the \C{/}
separator is the following one:
$$
\N{clear-switch}\ |\ [\C{@}]\N{ident}\ |\ \C{(}[\C{@}]\N{ident}\ \C{:=}\ \N{c-pattern}\C{)}
$$
These clear operations are intertwined with the generalization ones, which
helps in particular avoiding dependency issues while generalizing some facts.

\noindent
If an $\N{ident}$ is prefixed with \C{@} then its body (if any) is kept as
a let-in.  The syntax $\C{(}\N{ident}\C{:=}\N{c-pattern}\C{)}$ lets one
generalize an arbitrary term under a given name.  Note that the simplest
form \C{(x := y)} morally renames \C{y} to \C{x}; in this way one can
generalize over a section variable, since renaming does not require
the original variable to be cleared.

\noindent
The syntax \C{(@x := y)} generates a let-in abstraction but with the following
caveat: \C{x} will not bind \C{y}, but its body, whenever \C{y} can be
unfolded (i.e. not only in the case of a local definition, but also of a
global one).  Example:

\begin{lstlisting}
Section Test.
Variable x : nat.
Definition addx y := y + x.
Lemma test : x <= addx x.
wlog H : (y := x) (@twoy := addx x) / twoy = 2 * y.
\end{lstlisting}
\noindent
The first subgoal is:
\begin{lstlisting}
   (forall y : nat, let twoy := y + y in twoy = 2 * y -> y <= twoy) ->
   x <= addx x
\end{lstlisting}
\noindent
To avoid unfolding the term captured by the pattern \C{add x} one
can use the pattern \C{id (addx x)}, that would produce the following first
subgoal:
\begin{lstlisting}
   (forall y : nat, let twoy := addx y in twoy = 2 * y -> y <= twoy) ->
   x <= addx x
\end{lstlisting}
