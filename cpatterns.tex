\section{Contextual patterns}
\label{ssec:rewp}

The simple form of patterns used so far, $\N{term}s$ possibly containing
wild cards, often require an additional $\N{occ-switch}$ to be specified.
While this may work pretty fine for small goals, the use of polymorphic
functions and dependent types may lead to an invisible duplication of functions
arguments. These copies usually end up in types hidden by the implicit
arguments machinery or by user defined notations.  In these situations
computing the right occurrence numbers is very tedious because they must be
counted on the goal as printed after setting the \C{Printing All} flag.
Moreover the resulting script is not really informative for the reader, since
it refers to occurrence numbers he cannot easily see.

Contextual patterns mitigate these issues allowing to specify occurrences
according to the context they occur in.

\subsection{Syntax}

The following table summarizes the full syntax of
$\N{c-pattern}$ and the corresponding subterm(s) identified
by the pattern.
In the third column we use s.m.r. for
``the subterms matching the redex'' specified in the second column.

\begin{center}
\begin{tabularx}{\textwidth}{>{\arraybackslash}m{0.30\textwidth}|>{\arraybackslash}m{0.21\textwidth}|>{\arraybackslash}m{0.39\textwidth}}
$\N{c-pattern}$ & redex & subterms affected \\
\hline
$\N{term}$ & $\N{term}$ & all occurrences of $\N{term}$\\
\hline
$\N{ident}\ \C{in}\ \N{term}$ &
  subterm of $\N{term}$ selected by $\N{ident}$ &
  all the subterms identified by $\N{ident}$ in all
  the occurrences of $\N{term}$ \\
\hline
$\N{term}_1\ \C{in}\ \N{ident}\ \C{in}\ \N{term}_2$ & $\N{term}_1$ &
  in all  s.m.r. in all the subterms identified by $\N{ident}$ in all
  the occurrences of $\N{term}_2$ \\
\hline
$\N{term}_1\ \C{as}\ \N{ident}\ \C{in}\ \N{term}_2$ & $\N{term}_1$ &
  in all the subterms identified by $\N{ident}$ in all
  the occurrences of $\N{term}_2[\N{term}_1/\N{ident}]$\\
\hline
\end{tabularx}
\end{center}

The \C{rewrite} tactic supports two more patterns obtained
prefixing the first two with \C{in}. The intended meaning is that the
pattern identifies all subterms of the specified context. The
\C{rewrite} tactic will infer a pattern for the redex looking at the
rule used for rewriting.

\begin{center}
\begin{tabularx}{\textwidth}{>{\arraybackslash}m{0.30\textwidth}|>{\arraybackslash}m{0.21\textwidth}|>{\arraybackslash}m{0.39\textwidth}}
$\N{r-pattern}$ & redex & subterms affected \\
\hline
$\C{in}\ \N{term}$ & inferred from rule &
  in all  s.m.r. in all occurrences of $\N{term}$\\
\hline
$\C{in}\ \N{ident}\ \C{in}\ \N{term}$ & inferred from rule &
  in all  s.m.r. in all the subterms identified by $\N{ident}$ in all
  the occurrences of $\N{term}$ \\
\hline
\end{tabularx}
\end{center}

The first $\N{c-pattern}$ is the simplest form matching any
context but selecting a specific redex and has been described in the
previous sections. We have seen so far that the possibility of
selecting a redex using a term with holes is already a powerful mean of redex
selection. Similarly, any $\N{term}$s provided by the
user in the more complex forms of $\N{c-pattern}$s presented in the
tables above can contain holes.

For a quick glance at what can be expressed with the last
$\N{r-pattern}$ consider the goal \C{a = b} and the tactic
\begin{lstlisting}
  rewrite [in X in _ = X]rule.
\end{lstlisting}
It rewrites all occurrences of the left hand side of \C{rule} inside
\C{b} only (\C{a}, and the hidden type of the equality, are ignored).
Note that the variant \C{rewrite [X in _ = X]rule} would have
rewritten \C{b} exactly (i.e., it would only work if \C{b} and the
left hand side of \C{rule} can be unified).

\subsection{Matching contextual patterns}

The $\N{c-pattern}$s and $\N{r-pattern}$s involving
$\N{term}$s with holes are matched
against the goal in order to find a closed instantiation. This
matching proceeds as follows:

\begin{center}
\begin{tabularx}{\textwidth}{>{\arraybackslash}m{0.30\textwidth}|>{\arraybackslash}m{0.65\textwidth}}
$\N{c-pattern}$ & instantiation order and place for $\N{term}_i$ and redex\\
\hline
$\N{term}$ & $\N{term}$ is matched against the goal, redex is unified with
    the instantiation of $\N{term}$\\
\hline
$\N{ident}\ \C{in}\ \N{term}$ &
    $\N{term}$ is matched against the goal, redex is
    unified with the subterm of the
    instantiation of $\N{term}$ identified by $\N{ident}$\\
\hline
$\N{term}_1\ \C{in}\ \N{ident}\ \C{in}\ \N{term}_2$ &
    $\N{term}_2$ is matched against the goal, $\N{term}_1$ is
    matched against the subterm of the
    instantiation of $\N{term}_1$ identified by $\N{ident}$,
    redex is unified with the instantiation of $\N{term}_1$\\
\hline
$\N{term}_1\ \C{as}\ \N{ident}\ \C{in}\ \N{term}_2$ &
    $\N{term}_2[\N{term}_1/\N{ident}]$
    is matched against the goal,
    redex is unified with the instantiation of $\N{term}_1$\\
\hline
\end{tabularx}
\end{center}

In the following patterns, the redex is intended to be inferred from the
rewrite rule.

\begin{center}
\begin{tabularx}{\textwidth}{>{\arraybackslash}m{0.30\textwidth}|>{\arraybackslash}m{0.65\textwidth}}
$\N{r-pattern}$ & instantiation order and place for $\N{term}_i$ and redex\\
\hline
$\C{in}\ \N{ident}\ \C{in}\ \N{term}$ &
    $\N{term}$ is matched against the goal, the redex is
    matched against the subterm of the
    instantiation of $\N{term}$ identified by $\N{ident}$\\
\hline
$\C{in}\ \N{term}$ & $\N{term}$ is matched against the goal, redex is
   matched against the instantiation of $\N{term}$\\
\hline
\end{tabularx}
\end{center}

\subsection{Examples}

\subsubsection{Contextual pattern in \C{set} and the \C{:} tactical}

As already mentioned in section~\ref{ssec:set} the \C{set} tactic
takes as an argument a term in open syntax. This term is interpreted
as the simplest for of $\N{c-pattern}$. To void confusion in the grammar,
open syntax is supported only for the simplest form of patterns, while
 parentheses are required around more complex patterns.

\begin{lstlisting}
set t := (X in _ = X).
set t := (a + _ in X in _ = X).
\end{lstlisting}

Given the goal \C{a + b + 1 = b + (a + 1)} the first tactic
captures \C{b + (a + 1)}, while the latter \C{a + 1}.

Since the user may define an infix notation for \C{in} the former
tactic may result ambiguous. The disambiguation rule implemented is
to prefer patterns over simple terms, but to interpret a pattern with
double parentheses as a simple term. For example
the following tactic would capture any occurrence of the term `\C{a in A}'.

\begin{lstlisting}
set t := ((a in A)).
\end{lstlisting}

Contextual pattern can also be used as arguments of the \C{:} tactical.
For example:
\begin{lstlisting}
elim: n (n in _ = n) (refl_equal n).
\end{lstlisting}

\subsubsection{Contextual patterns in \C{rewrite}}
As a more comprehensive example consider the following goal:
\begin{lstlisting}
    (x.+1 + y) + f (x.+1 + y) (z + (x + y).+1) = 0
\end{lstlisting}
The tactic \C{rewrite [in f _ _]addSn} turns it into:
\begin{lstlisting}
    (x.+1 + y) + f (x + y).+1 (z + (x + y).+1) = 0
\end{lstlisting}
since the simplification rule \C{addSn} is applied only under the \C{f} symbol.
Then we simplify also the first addition and expand \C{0} into \C{0+0}.
\begin{lstlisting}
    rewrite addSn -[X in _ = X]addn0.
\end{lstlisting}
obtaining:
\begin{lstlisting}
    (x + y).+1 + f (x + y).+1 (z + (x + y).+1) = 0 + 0
\end{lstlisting}
Note that the right hand side of \C{addn0} is undetermined, but the
rewrite pattern specifies the redex explicitly. The right hand side of
\C{addn0} is unified with the term identified by \C{X}, \C{0} here.

The following pattern does not specify a redex, since it
identifies an entire region, hence the rewrite rule has to be instantiated
explicitly. Thus the tactic:
\begin{lstlisting}
    rewrite -{2}[in X in _ = X](addn0 0).
\end{lstlisting}
changes the goal as follows:
\begin{lstlisting}
    (x + y).+1 + f (x + y).+1 (z + (x + y).+1) = 0 + (0 + 0)
\end{lstlisting}
The following tactic is quite tricky:
\begin{lstlisting}
    rewrite [_.+1 in X in f _ X](addnC x.+1).
\end{lstlisting}
and the resulting goals is:
\begin{lstlisting}
    (x + y).+1 + f (x + y).+1 (z + (y + x.+1)) = 0 + (0 + 0)
\end{lstlisting}
The explicit redex \C{_.+1} is important since its head
constant \C{S} differs from the head constant inferred from
\C{(addnC x.+1)} (that is \C{addn}, denoted \C{+} here).
Moreover, the pattern \C{f _ X} is important to rule out the first occurrence
of \C{(x + y).+1}. Last, only the subterms of \C{f _ X} identified by \C{X} are
rewritten, thus the first argument of \C{f} is skipped too.
Also note the pattern \C{_.+1} is interpreted in the context
identified by \C{X}, thus it gets instantiated to \C{(y + x).+1} and
not \C{(x + y).+1}.

The last rewrite pattern allows to specify exactly the shape of the term
identified by \C{X}, that is thus unified with the left hand side of the
rewrite rule.
\begin{lstlisting}
    rewrite [x.+1 + y as X in f X _]addnC.
\end{lstlisting}
The resulting goal is:
\begin{lstlisting}
    (x + y).+1 + f (y + x.+1) (z + (y + x.+1)) = 0 + (0 + 0)
\end{lstlisting}

\subsection{Patterns for recurrent contexts}

The user can define shortcuts for recurrent contexts corresponding to the
\C{$\N{ident}$ in $\N{term}$} part. The notation scope identified
with \C{\%pattern} provides a special notation `\C{(X in t)}' the user
must adopt to define context shortcuts.

The following example is taken from \C{ssreflect.v} where the
\C{LHS} and \C{RHS} shortcuts are defined.

\begin{lstlisting}
Notation RHS := (X in _ = X)%pattern.
Notation LHS := (X in X = _)%pattern.
\end{lstlisting}

Shortcuts defined this way can be freely used in place of the
trailing \C{$\N{ident}$ in $\N{term}$} part of any contextual
pattern.
Some examples follow:

\begin{lstlisting}
set rhs := RHS.
rewrite [in RHS]rule.
case: (a + _ in RHS).
\end{lstlisting}
