\section{Gallina extensions}

Small-scale reflection makes an extensive use of the programming
subset of Gallina, \Coq{}'s logical specification language. This subset
is quite suited to the description of functions on representations,
because it closely follows the well-established design of the ML
programming language.The \ssr{} extension provides three additions
to Gallina, for pattern assignment, pattern testing, and polymorphism;
these mitigate minor but annoying discrepancies between Gallina and ML.

\subsection{Pattern assignment}\label{ssec:patass}
The \ssr{} extension provides the following construct for
irrefutable pattern matching, that is, destructuring assignment:
\begin{lstlisting}
  let: $\N{pattern}$ := $\N{term}_1$ in $\N{term}_2$
\end{lstlisting}
Note the colon `\L+:+' after the \C{let} keyword, which avoids any
ambiguity with a function
definition or \Coq{}'s basic destructuring \C{let}. The \C{let:}
construct differs from the latter in that
\begin{itemize}
\item The pattern can be nested (deep pattern matching), in
  particular, this allows expression of the form:
\begin{lstlisting}
 let: exist (x, y) p_xy := Hp in $\dots$
\end{lstlisting}
\item The destructured constructor is explicitly given in the
  pattern, and is used for type inference, e.g.,
\begin{lstlisting}
  Let f u := let: (m, n) := u in m + n.
\end{lstlisting}
using a colon \C{let:}, infers \C{f : nat * nat -> nat}, whereas
\begin{lstlisting}
  Let f u := let (m, n) := u in m + n.
\end{lstlisting}
with a usual \C{let}, requires an extra type annotation.
\end{itemize}
The \C{let:} construct is just (more legible) notation for the primitive Gallina expression
\begin{lstlisting}
  match $\N{term}_1$ with $\N{pattern}$ => $\N{term}_2$ end
\end{lstlisting}
Due to limitations of the \Coq{} v8 display API, a \C{let:} expression
will always be displayed with the \Coq{} v8.2 \L+let 'C $\dots$+ syntax
(see the \Coq{} documentation of the destructuring \C{let}
syntax). Unfortunately, this syntax does not handle user notation and
clashes with the lexical conventions of the \ssr{} library.
\todo{Here I really do not know how to say this.}

The \ssr{} destructuring assignment supports all the dependent match
annotations; the  full syntax is

\[ \C{let:} \N{pattern}_1\ \C{as}\ \N{ident}\ \C{in}\ \N{pattern}_2\ \C{:=}\ \N{term}_1\ \C{return}\ \N{term}_2\ \C{in}\ \N{term}_3\]

where $\N{pattern}_2$ is a \emph{type} pattern and $\N{term}_1$ and
$\N{term}_2$ are types.

When the \C{as} and \C{return} are both present, then $\N{ident}$ is bound
in both the type $\N{term}_2$ and the expression $\N{term}_3$;
variables in the optional type pattern $\N{pattern}_2$ are
bound only in the type $\N{term}_2$, and other variables in $\N{pattern}_1$ are
bound only in the expression $\N{term}_3$, however.

\subsection{Pattern conditional}\label{ssec:patcond}
The following construct can be used for a refutable pattern matching,
that is, pattern testing:
\[\C{if}\ \N{term}_1\ \C{is}\ \N{pattern}_1\ \C{then}\ \N{term}_2\ \C{else}
\ \N{term}_3\]
Although this construct is not strictly ML (it does exits in variants
such as the pattern calculus or the $\rho$-calculus), it turns out to be
very convenient for writing functions on representations,
because most such functions manipulate simple datatypes such as Peano
integers, options,
lists, or binary trees, and the pattern conditional above is almost
always the right construct
for analyzing such simple types. For example, the \C{|*null*|} and
\C{|*all*|} list function(al)s can be defined as follows:
\begin{lstlisting}
  Variable d: Set.
  Fixpoint |*null*| (s : list d) := if s is nil then true else false.
  Variable a : d -> bool.
  Fixpoint |*all*| (s : list d) : bool :=
     if s is cons x s' then a x && all s' else true.
\end{lstlisting}

The pattern conditional also provides a notation for destructuring
assignment with a refutable pattern, adapted to the pure functional
setting of Gallina, which lacks a \\\texttt{Match\_Failure} exception.

Like \C{let:} above, the \C{if$\dots$is} construct is just (more legible)
notation for the primitive Gallina expression:
\[\C{match}\ \N{term}_1\ \C{with}\ \N{pattern}\ \C{=>}\ \N{term}_2\ \C{| _ =>}\ \N{term}_3\ \C{end}\]
Similarly, it will always be displayed as the expansion of this form
in terms of primitive \C{match} expressions (where the default
expression $\N{term}_3$ may be replicated).


Explicit pattern testing also largely subsumes the generalization of
the \C{if} construct to all binary datatypes; compare:
\begin{lstlisting}
  if $\N{term}$ is inl _ then $\N{term}_l$ else $\N{term}_r$
\end{lstlisting}
and:
\begin{lstlisting}
  if $\N{term}$ then  $\N{term}_l$ else  $\N{term}_r$
\end{lstlisting}
The latter appears to be marginally shorter, but it is quite
ambiguous, and indeed often
requires an explicit annotation term : \C{{_}+{_}} to type-check,
which evens the character count.

Therefore, \ssr{} restricts by default the condition of a plain \C{if}
construct to the standard \C{bool} type; this avoids spurious type
annotations, e.g., in:
\begin{lstlisting}
  Definition |*orb*| b1 b2 := if b1 then true else b2.
\end{lstlisting}
As pointed out in section 1.2, this restriction can be removed with
the command:
\begin{lstlisting}
  Close Scope boolean_if_scope.
\end{lstlisting}
Like \C{let:} above, the \C{if $\ \N{term}\ $ is $\ \N{pattern}\ $
  else $\ \N{term}$} construct
supports the dependent \C{match} annotations:
\begin{lstlisting}
  if $\N{term}_1$ is $\N{pattern}_1$ as $\N{ident}$ in $\N{pattern}_2$ return $\N{term}_2$
    then $\N{term}_3$
    else $\N{term}_4$
\end{lstlisting}
As in \C{let:} the variable $\N{ident}$ (and those in
the type pattern $\N{pattern}_2$) are bound in $\N{term}_2$; $\N{ident}$ is
also bound in $\N{term}_3$ (but not in $\N{term}_4$), while the
variables in $\N{pattern}_1$ are bound only in $\N{term}_3$.

\noindent
Another variant allows to treat the else case first:
\[\C{if}\ \N{term}_1\ \C{isn't }\N{pattern}_1\ \C{then}\ \N{term}_2\ \C{else}
\ \N{term}_3\]
Note that $\N{pattern}_1$ eventually binds variables in $\N{term}_3$
and not $\N{term}_2$.

\subsection{Parametric polymorphism}\label{ssec:parampoly}
Unlike ML, polymorphism in core Gallina is explicit: the type
parameters of polymorphic  functions must be declared explicitly, and
supplied at each point of use. However, \Coq{} provides two features
to suppress redundant parameters:
\begin{itemize}
\item Sections are used to provide (possibly implicit) parameters for
  a set of definitions.
\item Implicit arguments declarations are used to tell \Coq{} to use
  type inference to deduce some parameters from the context at each
  point of call.
\end{itemize}
The combination of these features provides a fairly good emulation of ML-style
polymorphism, but unfortunately this emulation breaks down for
higher-order programming. Implicit arguments are indeed not inferred
at all points of use, but only at
points of call, leading to  expressions such as
\begin{lstlisting}
  Definition |*all_null*| (s : list T) := all (@null T) s.
\end{lstlisting}
Unfortunately, such higher-order expressions are quite frequent in
representation functions, especially those which use \Coq{}'s
\C{Structure}s to emulate Haskell type classes.

Therefore, \ssr{} provides a variant of \Coq{}'s implicit argument
declaration, which  causes \Coq{} to fill in some implicit parameters
at each point of use, e.g., the above definition can be written:
\begin{lstlisting}
  Definition |*all_null*| (s : list d) := all null s.
\end{lstlisting}
Better yet, it can be omitted entirely, since \C{all_null s} isn't
much of an improvement over \C{all null s}.

The syntax of the new declaration is
\begin{lstlisting}
  Prenex Implicits $\N{ident}^+$.
\end{lstlisting}
Let us denote $const_1 \dots const_n$ the list of identifiers given to a
\L+Prenex Implicits+ command.
The command checks that each \C{/*const$_i$*/}  is the name of a functional
constant, whose implicit arguments are prenex, i.e., the first $n_i >
0$ arguments of \C{/*const$_i$*/} are implicit; then it assigns
\L+Maximal Implicit+ status to these arguments.

As these prenex implicit arguments are ubiquitous and have often large
display strings, it is strongly recommended to change the default
display settings of \Coq{} so that they are not printed (except after a
\C{$\texttt{\textcolor{dkviolet}{Set }}$ Printing All} command).
All \ssr{} library files thus start with the incantation
\begin{lstlisting}
  Set Implicit Arguments.
  Unset Strict Implicit.
  Unset Printing Implicit Defensive.
\end{lstlisting}

\subsection{Anonymous arguments}
When in a definition, the type of a certain argument is mandatory, but
not its name, one usually use ``arrow'' abstractions for prenex
arguments, or the \L+(_ : $\N{term}$)+ syntax for inner arguments.
In \ssr{}, the latter can be replaced by the open syntax `\L+of $\N{term}$+'
or (equivalently)  `\L+& $\N{term}$+', which are both syntactically
equivalent to a \L+(_ : $\N{term}$)+ expression.

For instance, the usual two-contrsuctor polymorphic type \L+list+,
i.e. the one of the
standard {\tt List} library, can be defined by the following
declaration:
\begin{lstlisting}
  Inductive list (A : Type) : Type := nil | cons of A & list A.
\end{lstlisting}

\subsection{Wildcards}\label{ssec:wild}

The terms passed as arguments
to \ssr{} tactics can contain \emph{holes}, materialized by wildcards
\C{_}.
Since \ssr{} allows a more powerful form of type inference for these
arguments, it enhances the possibilities of using such wildcards.
These holes are in particular used as a convenient shorthand for
abstractions, especially in local definitions or type expressions.

Wildcards may be interpreted as abstractions (see for example sections
\ref{ssec:pose} and \ref{ssec:struct}), or their content can be
inferred from the whole
context of the goal (see for example section \ref{ssec:set}).
