\section{Introduction}\label{sec:intro}

This chapter describes an extension to the \Coq{} tactic language,
orginally designed to provide support at the tactic level for the
so-called \emph{small scale reflection} proof methodology.
\todo{Provide a citation here?} However, its tactics and tacticals
also provide features of general interest for a wide audience of users
like:
\begin{itemize}
\item better support for forward steps
\item common support for bookkeeping in all tactics
\item common support for subterm selection in all tactics
\item a unified interface for rewriting, definition expansion, and
  partial evaluation
\item support for so-called reflection steps.
\end{itemize}


In fact, only the last functionality is specific to small-scale
reflection. All the others are of general use. Moreover, these
features are implementing by extending the functionalities of existing
tactic, so that in the end proof scripts shall only feature
combinations of a very small number tactics with their intro-patterns
and switches.
\todo{Say something about the possibility to use a few features only?}

\subsection*{How to read this documentation}


The syntax of the tactics is presented as follows:
\begin{itemize}
\item \L+terminals+ are in typewriter font and $\N{non terminals}$ are
  between angle brackets.
\item Optional parts of the grammar are surrounded by $[\ ]$
  brackets. These should not be confused with verbatim brackets
  \L+[ ]+, which are delimiters in the \ssr{} syntax.
\item A vertical rule $|$ indicates an alternative in the syntax, and
  should not be confused with a
  verbatim vertical rule between verbatim brackets \L+[ | ]+.
\item A non empty list of non terminals (at least one item should be
  present) is represented by $\N{non terminals}^+$. A possibly empty
  one is represented by $\N{non terminals}^*$.
\item In a non empty list of non terminals, items are separated by blanks.
\end{itemize}


\noindent We follow the default color scheme of the \ssr{} mode for
ProofGeneral provided in the distribution:

\centerline{
\textcolor{dkblue}{\texttt{tactic}} or \textcolor{dkviolet}{\tt
  Command} or \textcolor{dkgreen}{\tt keyword} or
\textcolor{dkpink}{\tt tactical}}

\noindent Closing tactics/tacticals like \L+exact+ or \L+by+ (see section
\ref{ssec:termin}) are in red.

\subsection*{Acknowledgments}
The authors would like to thank Frédéric Blanqui, François Pottier
and Laurence Rideau for their comments and suggestions.

\newpage