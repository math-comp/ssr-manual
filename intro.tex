\section{Introduction}\label{sec:intro}
Small-scale reflection is a formal proof methodology based on the
pervasive use of computation with symbolic representations. Symbolic
representations are usually hidden in traditional computational
reflection (e.g., as used in the
\Coq{}\footnote{\url{http://coq.inria.fr}} \C{ring}, or \C{romega}):
they are generated on-the-fly by some heuristic algorithm and directly
fed to some decision or simplification procedure whose output is
translated back to "logical" form before being displayed to the
user. By contrast, in small-scale reflection symbolic representations
are ubiquitous; the statements of many top-level lemmas, and of most
proof subgoals, explicitly contain symbolic representations;
translation between logical and symbolic representations is performed
under the explicit, fine-grained control of the proof script.

The efficiency of small-scale reflection hinges on the fact that
fixing a particular symbolic representation strongly directs the
behavior of a theorem-prover:

\begin{itemize}
\item Logical case analysis is done by
enumerating the symbols according to their inductive type: the
representation describes which cases should be considered.
\item Many
logical functions and predicates are represented by concrete
functions on the symbolic representation, which can be computed once
(part of) the symbolic representation of objects is known: the
representation describes what should be done in each case.
\end{itemize}

Thus by controlling the representation we also control the automated
behavior of the theorem prover, which can be quite complex,
for example if a predicate is represented by a sophisticated decision
procedure. The real strength of small-scale reflection, however, is
that even very simple representations provide useful procedures. For
example, the truth-table representation of connectives, evaluated
left-to-right
on the Boolean representation of propositions, provides sufficient
automation for most propositional reasoning.

Small-scale reflection defines a basis for dividing the proof workload
between the user and the prover: the prover engine provides computation
and database functions (via partial evaluation, and definition and
type lookup, respectively), and the user script guides the execution
of these functions, step by step. User scripts comprise three kinds of
steps:

\begin{itemize}
\item Deduction steps directly specify part of the construction of
the proof, either top down (so-called forward steps), or bottom-up
(backward steps). A reflection step that switches between logical and
symbolic representations is just a special kind of deductive step.
\item
Bookkeeping steps manage the proof context, introducing, renaming,
discharging, or splitting constants and assumptions. Case-splitting on
symbolic representations is an efficient way to drive the prover
engine, because most of the data required for the splitting can be
retrieved from the representation type, and because specializing a
single representation often triggers the evaluation of several
representation functions.
\item Rewriting steps use equations to locally
change parts of the goal or assumptions.  Rewriting is often used to
complement partial evaluation, bypassing unknown parameters (e.g.,
simplifying \C{b && false} to \C{false}). Obviously, it's also used to
implement equational reasoning at the logical level, for instance,
switching to a different representation.
\end{itemize}

It is a characteristic of the
small-scale reflection style that the three kinds of steps are roughly
equinumerous, and interleaved; there are strong reasons for this,
chief among them the fact that goals and contexts tend to grow rapidly
through the partial evaluation of representations.  This makes it
impractical to embed most intermediate goals in the proof script - the
so-called declarative style of proof, which hinges on the exclusive
use of forward steps. This also means that subterm selection,
especially in rewriting, is often an issue.

The basic \Coq{} tactic language is not well adapted to small-scale
reflection proofs. It is heavily biased towards backward steps, with
little support for forward steps, or even script layout (these are
deferred to the "vernacular", i.e., \C{Section}/\C{Module} layer of
the input language). The support for rewriting is primitive, requiring
a separate tactic for each kind of basic step, and the behavior of
subterm selection is undocumented. Many of the basic tactics, such as
\C{intros}, \C{induction} and \C{inversion}, implement fragile context
manipulation heuristics which hinder precise bookkeeping; on the other
hand the under-utilized "intro patterns" provide excellent support for
case splitting.

The extensions presented here were designed to improve the
functionality of \Coq{} in all those areas, providing:

\begin{itemize}
\item support for better script layout
\item better support for forward steps
\item common
support for bookkeeping in all tactics
\item common support for subterm
selection in all tactics
\item a unified interface for rewriting,
definition expansion, and partial evaluation
\item improved robustness
with respect to evaluation and conversion
\item support for reflection
steps.
\end{itemize}
 We should point out that only the last functionality is specific
to small-scale reflection. All the others are of general use.
Moreover most of these features are introduced not
by adding new tactics, but by extending the functionality of existing
ones: indeed we introduce only three new tactics, rename three others,
but all subsume more than a dozen of the basic \Coq{} tactics.


\subsection*{How to read this documentation}


The syntax of the tactics is presented as follows:
\begin{itemize}
\item \L+terminals+ are in typewriter font and $\N{non terminals}$ are
  between angle brackets.
\item Optional parts of the grammar are surrounded by $[\ ]$
  brackets. These should not be confused with verbatim brackets
  \L+[ ]+, which are delimiters in the \ssr{} syntax.
\item A vertical rule $|$ indicates an alternative in the syntax, and
  should not be confused with a
  verbatim vertical rule between verbatim brackets \L+[ | ]+.
\item A non empty list of non terminals (at least one item should be
  present) is represented by $\N{non terminals}^+$. A possibly empty
  one is represented by $\N{non terminals}^*$.
\item In a non empty list of non terminals, items are separated by blanks.
\end{itemize}


\noindent We follow the default color scheme of the \ssr{} mode for
ProofGeneral provided in the distribution:

\centerline{
\textcolor{dkblue}{\texttt{tactic}} or \textcolor{dkviolet}{\tt
  Command} or \textcolor{dkgreen}{\tt keyword} or
\textcolor{dkpink}{\tt tactical}}

\noindent Closing tactics/tacticals like \L+exact+ or \L+by+ (see section
\ref{ssec:termin}) are in red.

\newpage