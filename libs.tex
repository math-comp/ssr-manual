\section{Using the libraries}

This section contains a short description of the libraries
present in the Ssreflect and Mathematical Components
distributions. For further information, please
refer to the comments included in the {\tt .v} source files of the
distributed libraries.


\subsection{Some general design choices}

\subsubsection*{Use of CoInductive types}

Coinductive types declared throughout the libraries should be
considered as data structures enjoying case analysis but not induction.
This design choice is meant
to stop the user from doing meaningless induction rather than
to represent infinite objects.

\subsubsection*{Boolean propositions, support for small scale reflection}

The {\tt ssreflect} library
provides basic support for the view mechanism, together with some
technical lemmas and definitions, dedicated to the
internal mechanism of some \ssr{} tactics, and to the reduction tags
(\L+nosimpl+, \L+lock+, see section \ref{ssec:lock}). This {\tt
  ssreflect.v} file being devoted to technical infrastructure, like
 {\tt ssreflect.ml}, it does not contain much insight on the
 constructions provided.



The {\tt ssrbool} library extends the possibilities of the view
mechanism by providing the support needed for small scale reflection.

The main ingredients of this support are:
\begin{itemize}
\item The \L+|*is_true*| : bool >-> Prop+ coercion;
\item The definition of the \L+|*reflect*|+ predicate (see
section \ref{ssec:reflpred}), together with the:
\begin{lstlisting}
  Coercion |*elimT*| : reflect >-> Funclass
\end{lstlisting}
which makes possible to apply directly a reflection lemma to a boolean
assertion;
\item A bunch of reflection lemmas, which directly reflect
  boolean connectives into their logical counterparts (like \L+|*andP*|+),
  or propagate negations (like \L+|*norP*|+);
\item Predefined view hints, which help the reflection mechanism
  handle complex boolean statements to be reflected into logical ones
  (like \L+|*introNTF*|+).
\end{itemize}

The {\tt ssrbool} library also contains several tool-kits for handling
boolean predicates that become pervasive in a boolean small scale
reflection setting, like :
\begin{itemize}
\item Properties of boolean connectives, including the
  various commutativity, associativity, distributivity (rewritable)
  properties and a \L+bool_congr+ heuristic for weeding out common
  terms from boolean equalities;
\item Iterated boolean operations, comprising some convenient n-ary
  ``list-style'' notations for these kind of predicates, in the form:

\centerline{\L+[op arg1, arg2, ... last_separator last_arg]+}

\noindent See the comments in {\tt ssrbool.v} for more details.
\item Support for applicative and collective predicate, where
  collective predicates deserve infix notation like \L+(x \ in p)+;
\item Properties and predicates on boolean relations and the localized

\end{itemize}

For more details, please refer to the comments in {\tt ssrbool}.

\subsubsection*{Canonical Structures}

The use of \L+Canonical Structure+s is pervasive in \ssr{} libraries.
This provides in particular a mechanism of proof inference, used as a
Prolog-like proof inference engine as often as possible.

For instance, once the following irrelevance
result on \L+eqType+ structures (see {\tt eqtype.v}) is proved:
\begin{lstlisting}
  Theorem |*eq_irrelevance*| (T : eqType) (x y : T) (e1 e2 : x = y) : e1 = e2.
\end{lstlisting}
it is instantiated for booleans (see again \texttt{eqtype.v}), and
proved in this way:
\begin{lstlisting}
  Lemma |*bool_irrelevance*| : forall (x y : bool) (E E' : x = y), E = E'.
  Proof. exact: eq_irrelevance. Qed.
\end{lstlisting}
Notice that the \L+eqType+ structure of booleans, which is required to
apply the \L+eq_irrelevance+ theorem, is automatically
inferred by the system from the \L+bool+ type of \L+x+ and \L+y+.

Notice also that the tactic:
\begin{lstlisting}
  exact eq_irrelevance.
\end{lstlisting}
(without the '\L+:+ tactical) would fail closing the goal because
the unification algorithm used by the standard \L+exact+
tactic does not take \L+Canonical Structure+ equations into
account. On the other hand, the standard \L+refine+ and \L+apply+
tactics  know about these equations (with standard \Coq{} version
$\geq$ 8.4 ).
%  This is why the combinations
% \L+apply:+ and \L+exact:+ make the use of \L+Canonical Structure+s
% really tractable, when this feature of standard \Coq{} is convenient
% to use in practice with standard 8.4 \Coq{}.

In fact, \Coq{}'s \L+Canonical Structure+ mechanism pre-computes some
projection equations that are later used as hints during
unification. Canonical structures are hence inferred automatically
when a statement explicitly involves projections \emph{values}.
In contrast, canonical structure inference cannot directly unify the
structure \emph{type} with that of its projections, despite the fact
that \L+Coercion+s may give the illusion that this happens.

For example, the following script will be accepted
\begin{lstlisting}
  Require Import ssreflect eqtype.
  Variables A B C : eqType.
  Hypothesis BtimesCisA : (B * C = A)%type.
\end{lstlisting}
because type-checking will insert the \L+Equality.sort+ coercion projection
to \L+A+ in the right-hand side of the equation. However,
\begin{lstlisting}
  Hypothesis AisBtimesC : (A = B * C)%type.
\end{lstlisting}
raises the error message: \texttt{Error: The term "(B *
  C)\%type" has type "Type" while it is expected to have type "eqType"}
even though the \texttt{eqtype} library contains a canonical construction
of the eqType structure of the Cartesian product of two eqType
structures: unlike canonical projections, coercions only work one way.

This illustrates the need for a way of directly looking up a canonical
structure. This operation is supported, with the general syntax:
$$\C{[}struct\_type\ \C{of}\ carrier\_type\ [\C{for}\ t\ ]\C{]}$$
This computes the canonical structure of type
$struct\_type$ declared up to conversion on the type $carrier\_type$.
The structure inferred should be convertible to the optional argument
$t$.

Going back to the previous example:
\begin{lstlisting}
  Hypothesis AisBtimesC : A = [eqType of (B * C)%type].
\end{lstlisting}
is accepted by the system.


Some types are defined by requiring a structure on their
parameter. For instance, the type \L+polynomial+ of polynomials (in
{\tt poly.v} in the Mathematical Components distribution) has one
parameter, which should be equipped with a ring structure. In that
case, a curly bracket notation is provided to allow \Coq{} to infer said
structure rather that having to supply it explicitly. This notation
should always be preferred to the type itself. Hence one should always
use \L+{poly R}+, rather than \L+(polynomial R_ringType)+, when
\L+R_ringType+ would be the ring structure with carrier \L+R+. If this
\L+R_ringType+ structure is canonical, it will automatically be
inferred in \L+{poly R}+.

\subsection{\ssr{} proving guidelines}

Here are a few tips to write proofs using \ssr{}.

\subsubsection*{Proof formatting guidelines}

\begin{itemize}
\item {\bf Loaded libraries.} The order matters! For instance, {\tt
    bigop} should be loaded before {\tt ssralg}, to make the
  notations work.

\item {\bf Delimiters.} A space should always follow a delimiter
  symbol, and spaces should surround operator symbols. Similarly,
  spaces should always surround a type casting colon.

\item {\bf How to write pairs.} A tuple is parenthesized and the
  commas therein (delimiters) are each followed by a space:
  \L+(1, 2)+.

\item {\bf How to write sequences.} Write \L+ x :: l+ with spaces around
  the \L+::+ (since :: is an infix operator, hence surrounded by
  spaces) and \L+[:: x1; x2; x3]+ or \L+[:: x1, x2 & tl]+ (since \L+,+
  is a delimiter, hence followed by a space).

\item  {\bf How to write auto-simplifying functions.} Auto-simplifying
  functions should be defined using the bracket notation provided by
  the \texttt{ssrfun} library. The expression \L+[fun x y => f x y]+
  denotes the auto-simplifying version of the binary function \L+f+.

\item {\bf How to write auto-simplifying predicates.} Following the
  bracket notation for auto-simplifying functions, the \texttt{ssrbool}
  library provides a
  comprehension-style notation for auto-simplifying boolean predicates,
  in the form: \texttt{[type var separator expr]}. For instance
  \L+[pred x | x == y]+ is the auto-simplifying boolean comparison to \L+y+.


\item {\bf How to write composed predicates.} The \texttt{ssrbool}
  library
  provides sequence-like notations for the
  iteration of a right associative operator on a list of
  predicates (boolean or not).
  For instance \L+[/\ P1, P2 & P3]+ (resp. \L+[\/ P1, P2 | P3]+)
  denotes \L+P1 /\ P2 /\ P3+ (resp. \L+P1 \/ P2 \/ P3+). See comments
  and reserved notations in \texttt{ssrbool.v} for the complete list of
   available notations.


\item {\bf How to write operator symbols.} For sake of readability,
  operator symbols should
  be separated from their arguments by spaces, like in \L+x * y+ or
  \texttt{a (+)  b}.
  Perhaps more importantly,  this should prevent
  \Coq{} from confusing them with multi-characters notations.

\item {\bf How to write a partial application w.r.t. the {\it second}
    argument.}
The {\tt fun } library provides a convenient ``placeholder'' notation
for such abstractions:
\begin{lstlisting}
  Notation "f ^~ y" := (fun x => f x y)
\end{lstlisting}

\item {\bf How to cast an equality.} To force the application of an
  existing coercion on the left hand side of a Leibniz equality, and
  hence in the type provided to \L+eq+, use the \L+:>+ cast. For
  instance \L+(true = 1)+ is not type-checked, but \L+(true = 1 :> nat)+
  is\footnote{The {\tt ssrnat} library defines such a coercion from
    {\tt bool} to {\tt nat}.}. Note that this feature is provided by
  the standard \Coq{}.

\item {\bf How to state standard properties on operators.} Use the
  predicates defined in the {\bf ssrfun} library :
  \L+|*left_inverse*|+, \L+|*right_distributive*|+,
  \L+|*commutative*|+, etc to normalize these statements.


\end{itemize}

\subsubsection*{Layout of proofs}

\begin{itemize}
\item {\bf Width of the page.} The library has been developed using a
  80 characters wide page. Maintaining this convention in your
  developments improves the readability of scripts. This is specially
  relevant in \ssr{} scripts, where tactics are usually chained in
  longer lines than in most standard \Coq{} scripts.

\item{\bf Starting a proof.} Start each proof with
  \L+Proof+. If the scripts of the proof is a single line of tactics,
  then this line should start with \L+Proof.+ and finish with \L+Qed.+
  If it doesn't the first line of the script should only contain
  \L+Proof.+ and the last line should only contain \L+Qed.+.


\item {\bf Lines of tactics in a proof.} Tactics should be chained on
  the whole line using the semi-colon (or chained rewriting steps for
  instance), but a block between two points should never take more
  than one line. This often helps to have lines with a semantic unity.

  Start a new line each time you start a new subgoal
  proof, or use a forward chaining tactic, or state a local definition
  or an abbreviation.

\item{\bf Linearity of scripts.} Try to make your script as linear as
  possible. Only open an indented piece of script for a non trivial
  subgoal. To improve linearity use the closing switch \L+//+ and the
  optional rewrite switch \L+?+ as often as needed. The
  \L+first+ and \L+last+ selectors also help closing an easy goal on the
  same line as the tactic which had created it.

\item {\bf Indentation of proofs.} Use indentation to separate the
  scripts of two-cases proofs (like when using a forward
  chaining tactic). If the proof of one of the goals is short enough,
  use selectors (see section \ref{ssec:select}) to close it as soon
  as it is created. If a branching proof has more that two cases,
  use bullets (see section \ref{ssec:indent}) to separate the
  corresponding scripts.

\item{\bf Grabbing subterms in the goal. } Do not copy and paste large
  subterms from the current goal to the script. Most often patterns
  with wildcards
  will do the job for you, for instance by the mean of an abbreviation
  (see section \ref{ssec:set}).

\item {\bf Closing goals.} The last tactic of a script, which  closes
  the goal, should be a closing tactic. Remember any tactic is turned
  into a closing one by being preceded by the \L+by+ tactical. When
  the last tactic of a script takes a whole line of possibly chained
  tactics, the \L+by+ tactical should start this line.


\end{itemize}

\subsubsection*{Name Policy}

  Lemmas in the distributed libraries respect the following name policy:
\begin{itemize}
\item{\bf Generalities}
  \begin{itemize}
  \item Most of the time the name of a lemma can be read off its
    statement: a lemma named \L+|*fee_fie_foe*|+ will say something about
    \L+(fee .. (fie ..(foe ..) ..) ..)+, e.g. lemma \L+|*size_cat*|+ in
    {\tt seq.v}.
  \item We often use a one-letter suffix to resolve overloaded
    notation, e.g., addn, addb, addr denote nat, boolean, ring
    addition, respectively. This policy does not necessarily apply to
    constants that should always be hidden behind a generic notation,
    and handled by a more generic theory.
  \item Finally, a handful of theorems have a historical name,
    e.g, \L+|*Cayley_Hamilton*|+ or \L+|*factor_theorem*|+.
  \end{itemize}
\item{\bf Structures and Records}
  \begin{itemize}
  \item Each structure type starts with a
    lower case letter, and its constructor has the same name but with a
    capital first letter.
  \item Each instance of a structure type has a name formed with the
    name of the carrier type, followed by an underscore and the one of
    the structure type like in \L+seq_sub_subType+, the structure of
    \L+subType+ defined on \L+seq_sub+ (see {\tt fintype.v}). Notable
    exceptions to this rule are canonical constructions taking
    benefits of modular name spaces, like in {\tt ssralg.v}.
  \end{itemize}
\item {\bf Suffixes}
  \begin{itemize}
  \item Lemma whose conclusion is a predicate, or an equality
    for a predicate: that predicate is a suffix of the lemma name,
    like in \L+|*addn_eq0*|+ or \L+|*rev_uniq*|+.
  \item Lemmas whose conclusion is a standard property such as
    \L+\char+, \L+<|+, etc.\footnote{These examples are taken from
      libraries in the Mathematical Component distribution.}:
    the property should be
    indicated by a suffix (like \L+ _char+, \L+_normal+, etc), so
    the lemma name
    should start by a description of the argument of the property, such as
    its key property, or its head constant.
    Thus we have \L+|*quotient_normal*|+, not \L+normal_quotient+, etc. This
    convention does not apply to monotony rules, for which we either
    use the name of the property with the suffix for the operator
    (e.g., \L+|*groupM*|+), or the name of the operator with the S
    suffix for subset monotony (e.g., \L+|*mulgS*|+).
  \item We try to use and maintain the following set of lemma suffixes:
    \begin{itemize}
    \item {\tt 0} : zero, or the empty set
    \item {\tt 1} : unit, or the singleton set (use \L+_set1+ for
      the latter to disambiguate)
    \item {\tt 2} : two, doubling, doubletons
    \item {\tt 3} etc, similarly
    \item {\tt A} : associativity
    \item {\tt C} : commutativity, or set complement (use \L+Cr+
      for trailing complement)
    \item {\tt D} : set difference, addition
    \item {\tt E} : definition elimination (often conversion
      lemmas)
    \item {\tt F} : boolean false, finite type variant (as in
      \L+|*canF_eq*|+), or group functor
    \item {\tt G} : group argument
    \item {\tt I} : set intersection, injectivity for binary operators
    \item {\tt J} : group conjugation
    \item {\tt K} : cancellation lemmas
    \item {\tt L} : left hand side (in \L+|*canLR*|+)
    \item {\tt M} : group multiplication
    \item {\tt N} : boolean negation, additive opposite
    \item {\tt P} : characteristic properties (often reflection
      lemmas)
    \item {\tt R} : group commutator, or right hand side (in
      \L+|*canRL*|+)
    \item {\tt S} : subset argument, or integer successor
    \item {\tt T} : boolean truth and Type-wide sets
    \item {\tt U} : set union
    \item {\tt V} : group or ring multiplicative inverse
    \item {\tt W} : weakening
    \item {\tt X} : exponentiation, or set cartesian product
    \item {\tt Y} : group join
    \item {\tt Z} : module/vector space scaling
    \end{itemize}
  \end{itemize}
\end{itemize}
% \subsubsection*{Name Policy}

% In addition to the searching tool, the lemmas and
% constructors follow some conventions helping the user to guess their
% names when possible:
% \begin{itemize}
% \item Each structure type starts with a lower case letter, and its
%   constructor has
%   the same name but with a capital first letter;
% \item Each instance of a structure type has a name formed with the
%   name of the carrier type, followed by an underscore and the one of
%   the structure type. For instance, \L+bool_eqType+ is an \L+EqType+
%   structure of carrier \L+bool+.
% \item Properties of operators are prefixed with the name of
%   the operator and postfixed with a description of the property (like
%   \L+addnC+ for the commutativity of the \L+addn+ operations, \L+addnA+
%   for its associativity, \L+addnE+ for the elimination (of the static
%   \L+nosimpl+ tag) lemma...).
% \end{itemize}



\subsubsection*{Syntax for Gallina extensions}
\begin{itemize}
\item {\bf Irrefutable patterns.} Use the \ssr{} syntax
\L+let: ... in+ for irrefutable patterns (see section
\ref{ssec:patass}).
\item {\bf Type annotations for anonymous arguments.} Using the open
  \ssr{} syntax '\L+& $\N{term}$+' and '\L+of $\N{term}$+' instead of the
  standard  \L+(_ : $\N{term}$)+ like in:
\begin{lstlisting}
  Inductive list (A : Type) : Type := nil | cons of A & (list A).
\end{lstlisting}
tend to make type annotations in definitions much more readable.
\item{\bf Open syntax.} Several \ssr{} tactics (\L+pose+, \L+have+,...)
  support open syntax and applicative-style definitions for functions,
  which improves the readability of statements.
  It is hence recommended practice to use for instance:
\begin{lstlisting}
  pose f x y := x + y.
\end{lstlisting}
instead of the standard \Coq{} equivalent tactic:
\begin{lstlisting}
  pose f := (fun x y => x + y).
\end{lstlisting}

\end{itemize}

\subsection{Interfaces}

Figure \ref{fig:hier} summarizes the main interfaces available in
the present distribution, and their interdependencies. % Blue, starred
% boxes are dedicated to interfaces that would collapse either on a
% brown unstarred box or on \L+Type+ in an untyped, classical setting.

The definition of each of these interfaces obeys a systematic
pattern. However, unless the user wants to add a new structure to
this picture, it should not be necessary to delve into the details of
these definitions. On the other hand, this systematic way of defining
the interfaces leads to a systematic way of populating them. Please
refer to comments in the files for the rationale of each interface and
for a review of the constructions provided.

% Note that a {\tt coqdoc} snapshot of the distributed libraries,
% including a search tool, is browsable on-line at \url{http://coqfinitgroup.gforge.inria.fr/}.

\newpage

\begin{figure}[!h]
  \begin{center}
      \includegraphics[width=\linewidth]{Figs/classes1}
   \caption{Interfaces defined in the ssreflect
     libraries. Note that the {\tt finalg} library defines a clone of the ssralg
     sub-hierarchy for structures based on finite carriers.}\label{fig:hier}
\end{center}
\end{figure}
