\section{Rewriting}\label{sec:rw}
The generalized use of reflection implies that most of the
intermediate results handled are properties of effectively computable
functions. The most efficient mean of establishing such results are
computation and simplification of expressions involving such
functions, i.e., rewriting. We have therefore defined an extended
\C{rewrite} tactic that unifies and combines most of the rewriting
functionalities.
\subsection{An extended \C{rewrite} tactic}\label{ssec:extrw}
The main improvements brought to the standard \Coq{} \C{rewrite}
tactic are:
\begin{itemize}
\item Whereas the primitive \C{rewrite} tactic can only perform a
  single rewriting
operation in the goal or in the context, the extended \C{rewrite}
can perform an entire series of such operations in any subset of the
goal and/or context;
\item The \ssr{} \C{rewrite} tactic allows to perform rewriting,
  simplifications, folding/unfolding of definitions, closing of goals;
\item Several rewriting operations can be chained in a single tactic;
\item Control over the occurrence at which rewriting is to be performed is
  significantly enhanced.
\end{itemize}


The general form of an \ssr{} rewrite tactic is:
\begin{lstlisting}
   rewrite $\N{rstep}^+$.
\end{lstlisting}
The combination of a rewrite tactic with the \C{in} tactical (see
section \ref{ssec:loc}) performs rewriting in both the context and the
goal.

A rewrite step $\N{rstep}$ has the general form:
$$[\N{r-prefix}]\N{r-item}$$
where:

\begin{eqnarray*}
\N{r-prefix} & \equiv &
[\C{-}]\ [\N{mult}]\ [\N{occ-switch} | \N{clear-switch}]\ [\C{[}\N{r-pattern}\C{]}]\\
\N{r-pattern} & \equiv &
   \N{term}\ |\ \C{in}\ [\N{ident}\ \C{in}]\ \N{term} \
   |\ [\N{term}\ \C{in}\ |\ \N{term}\ \C{as}]\ \N{ident}\ \C{in}\ \N{term}\\
\N{r-item} & \equiv &
 [\C{/}]\N{term} \ |\ \N{s-item} \\
\end{eqnarray*}


% \begin{eqnarray*}
% \N{r-prefix} & \equiv &
% [\C{-}]\ [\N{mult}][\N{occ-switch} | \N{cl-item}][\N{term}]\\
% \N{r-item} & \equiv &
% [\C{-}]\N{term}\ |\ [\C{-}]\C{[}\N{term}_1\C{]}\C{/(}\N{term}_2\C{)} \ |\
% \N{simpl switch} \ |\  \\
% && \N{eq-term} \ |\  \C{(} \N{eq-term}_1\C{,}\dots
% \C{,}\N{eq-term}_n \C{)} \ |\  \C{(_ :}\N{eq-term} \C{)}
% \end{eqnarray*}


An $\N{r-prefix}$ contains annotations to qualify where and how   the
rewrite operation should be performed:
\begin{itemize}
\item The optional initial \L+-+ indicates the direction of the rewriting
  of $\N{r-item}$: if present the direction is right-to-left and it is
  left-to-right otherwise.
\item The multiplier $\N{mult}$ (see section \ref{ssec:iter})
  specifies if and how the rewrite operation should be repeated.
\item A rewrite operation matches the occurrences of a \emph{rewrite
  pattern}, and replaces these occurrences by an other term, according
  to the given $\N{r-item}$.
  The optional \emph{redex switch} $\C{[}\N{r-pattern}\C{]}$, which
  should always be surrounded by brackets, gives explicitly this
  rewrite pattern. In its simplest form, it is a regular term.
  If no explicit redex switch
  is present the rewrite pattern to be matched is inferred from the
  $\N{r-item}$.
\item This optional $\N{term}$, or
  the $\N{r-item}$, may be preceded by an
  occurrence switch (see section \ref{ssec:select}) or a clear item
  (see section \ref{ssec:discharge}), these two possibilities being
  exclusive. An occurrence switch selects the occurrences of the
  rewrite pattern which should be affected by the rewrite operation.
\end{itemize}


An $\N{r-item}$ can be:


\begin{itemize}
\item A \emph{simplification r-item}, represented by a
  $\N{s-item}$ (see section \ref{ssec:intro}).
% In some cases, $\N{r-prefix}$es are not supported.
  Simplification operations are
  intertwined with the possible other rewrite operations specified by
  the list of r-items.
\item A \emph{folding/unfolding r-item}. The tactic:
\begin{lstlisting}
  rewrite //*term*/
\end{lstlisting}
unfolds the head constant of \textit{term} in every occurrence of the
first matching of \textit{term} in the goal. In particular, if
\L+my_def+ is a (local or global) defined constant, the tactic:
\begin{lstlisting}
  rewrite /my_def.
\end{lstlisting}
is in principle\footnote{The implementation of these fold/unfold tactics does
  not call standard \Coq{} \C{fold} and \C{unfold}.} equivalent to:
\begin{lstlisting}
  unfold my_def.
\end{lstlisting}
Conversely:
\begin{lstlisting}
  rewrite -/my_def.
\end{lstlisting}
is equivalent to:
\begin{lstlisting}
  fold my_def.
\end{lstlisting}
%\emph{Warning} The combination of redex switch with unfold
%$\N{r-item}$ is not yet implemented.

When an unfold r-item is combined with a redex pattern, a conversion
operation is performed. A tactic of the form:
\begin{lstlisting}
  rewrite -[$\N{term}_1$]/$\N{term}_2$.
\end{lstlisting}
is equivalent to:
\begin{lstlisting}
  change $\N{term}_1$ with $\N{term}_2$.
\end{lstlisting}

If \C{$\N{term}_2$} is a single constant and \C{$\N{term}_1$} head symbol
is not \C{$\N{term}_2$}, then the head symbol of \C{$\N{term}_1$} is
repeatedly unfolded until \C{$\N{term}_2$} appears.

\begin{lstlisting}
  Definition double x := x + x.
  Definition ddouble x := double (double x).
  Lemma ex1 x : ddouble x = 4 * x.
  rewrite [ddouble _]/double.
\end{lstlisting}

The resulting goal is:

\begin{lstlisting}
  double x + double x = 4 * x
\end{lstlisting}

\emph{Warning} The \ssr{} terms containing holes are \emph{not}
typed as abstractions in this context. Hence the following script:
\begin{lstlisting}
  Definition f := fun x y => x + y.
  Goal forall x y, x +  y = f y x.
  move=> x y.
  rewrite -[f y]/(y + _).
\end{lstlisting}
raises the error message
\begin{verbatim}
   User error: fold pattern (y + _) does not match redex (f y)
\end{verbatim}
but the script obtained by replacing the last line with:
\begin{lstlisting}
  rewrite -[f y x]/(y + _).
\end{lstlisting}
is valid.


\item A term, which can be:
  \begin{itemize}
    \item A term whose type has the form:
      $$\C{forall}\ (x_1\ :\ A_1)\dots(x_n\ :\ A_n),\ eq\ term_1\ term_2$$
      where $eq$ is the Leibniz equality or a registered setoid
      equality. %In the case of setoid relations, the only supported
      %r-prefix is the directional \L+-+.
    \item A list of terms $(t_1,\dots,t_n)$, each $t_i$ having a type of the
      form: $$\C{forall}\ (x_1\ :\ A_1)\dots(x_n\ :\ A_n),\ eq\ term_1\ term_2$$ where
      $eq$ is the Leibniz equality or a registered setoid
      equality. The tactic:
      \begin{lstlisting}
        rewrite /*r-prefix*/ ($t_1$,$\dots$,$t_n$).
      \end{lstlisting}
      is equivalent to:
\begin{lstlisting}
  do [rewrite /*r-prefix*/ $t_1$ | $\dots$ | rewrite /*r-prefix*/ $t_n$].
\end{lstlisting}
    \item An anonymous rewrite lemma
      \L+(_ : /*term*/)+, where \textit{term} has again the form:
      $$\C{forall}\ (x_1\ :\ A_1)\dots(x_n\ :\ A_n),\ eq\ term_1\ term_2$$
      The tactic:
      \begin{lstlisting}
        rewrite (_ : /*term*/)
      \end{lstlisting}
      is in fact equivalent to the standard \Coq{}:
      \begin{lstlisting}
        cutrewrite (/*term*/).
      \end{lstlisting}

  \end{itemize}

\end{itemize}




\subsection{Remarks and examples}\label{ssec:rwex}

\subsubsection*{Rewrite redex selection}
The general strategy of \ssr{}
is to grasp as many redexes as possible and to let the user select the
ones to be rewritten thanks to the improved syntax for the control of
rewriting.

This may be a source of incompatibilities between \ssr{} and standard
\Coq{}.

In a rewrite tactic of the form:
\begin{lstlisting}
  rewrite $\N{occ-switch}$[$\N{term}_1$]$\N{term}_2$.
\end{lstlisting}
$\N{term}_1$ is the explicit rewrite redex and
$\N{term}_2$ is the
rewrite rule. This execution of this tactic unfolds as follows:

\begin{itemize}
\item First $\N{term}_1$ and $\N{term}_2$ are $\beta\iota$ normalized. Then
  $\N{term}_2$ is put in head normal form if the Leibniz equality
  constructor \C{eq} is not the head symbol. This may involve $\zeta$
  reductions.
\item Then, the matching algorithm (see section \ref{ssec:set})
  determines the first subterm of the goal matching the rewrite pattern.
  The rewrite pattern is
  given by $\N{term}_1$, if an explicit redex pattern switch is provided, or by
  the type of $\N{term}_2$ otherwise. However, matching skips over
  matches that would lead to trivial rewrites. All the
  occurrences of this subterm in the goal are candidates for rewriting.
\item Then only the occurrences coded by $\N{occ-switch}$ (see again
  section \ref{ssec:set}) are finally selected for rewriting.
\item The left hand side of $\N{term}_2$ is unified with the subterm found
  by the matching algorithm, and if this succeeds, all the selected
  occurrences in the goal are replaced by the right hand side of
  $\N{term}_2$.
\item Finally the goal is $\beta\iota$ normalized.
\end{itemize}

In the case $\N{term}_2$ is a list of terms, the first top-down (in
the goal) left-to-right (in the list) matching rule gets selected.

\subsubsection*{Chained rewrite steps}


The possibility to chain rewrite operations in a single tactic makes
scripts more compact and gathers in a single command line a bunch
of surgical
operations  which would be described by a one sentence in a pen and
paper proof.

Performing rewrite and simplification operations in a single tactic
enhances significantly the concision of scripts. For instance the
tactic:
\begin{lstlisting}
  rewrite /my_def {2}[f _]/= my_eq //=.
\end{lstlisting}
unfolds \C{my_def} in the goal, simplifies the second occurrence of the
first subterm matching pattern \L+[f _]+, rewrites \C{my_eq},
simplifies the whole goal and closes trivial goals.

Here are some concrete examples of chained rewrite operations, in the
proof of basic results on natural numbers arithmetic:

\begin{lstlisting}
  Lemma |*addnS*| : forall m n, m + n.+1 = (m + n).+1.
  Proof. by move=> m n; elim: m. Qed.

  Lemma |*addSnnS*| : forall m n, m.+1 + n = m + n.+1.
  Proof. move=> *; rewrite addnS; apply addSn. Qed.

  Lemma |*addnCA*| : forall m n p, m + (n + p) = n + (m + p).
  Proof. by move=> m n; elim: m => [|m Hrec] p; rewrite ?addSnnS -?addnS. Qed.

  Lemma |*addnC*| : forall m n, m + n = n + m.
  Proof. by move=> m n; rewrite -{1}[n]addn0 addnCA addn0. Qed.
\end{lstlisting}

Note the use of the \L+?+ switch for parallel rewrite operations in
the proof of \L+|*addnCA*|+.

\subsubsection*{Explicit redex switches are matched first}
If an $\N{r-prefix}$ involves a \emph{redex switch}, the first step is to
find a subterm matching this redex pattern, independently from the left hand
side \C{t1} of the equality the user wants to rewrite.

For instance, if \L-H : forall t u, t + u = u + t- is in the context of a
goal \L-x + y = y + x-, the tactic:
\begin{lstlisting}
  rewrite [y + _]H.
\end{lstlisting}
transforms the goal into \L-x + y = x + y-.

Note that if this first pattern matching is not compatible with the
\emph{r-item}, the rewrite fails, even if the goal contains a correct
redex matching both the redex switch and the left hand side of the
equality. For instance, if \L-H : forall t u, t + u * 0 = t- is
in the context of a goal \L-x + y * 4 + 2 * 0 = x + 2 * 0-, then tactic:
\begin{lstlisting}
  rewrite [x + _]H.
\end{lstlisting}
raises the error message:
\begin{verbatim}
  User error: rewrite rule H doesn't match redex (x + y * 4)
\end{verbatim}
while the tactic:
\begin{lstlisting}
  rewrite (H _ 2).
\end{lstlisting}
transforms the goal into \L-x + y * 4 = x + 2 * 0-.


\subsubsection*{Occurrence switches and redex switches}
The tactic:
\begin{lstlisting}
  rewrite {2}[_ + y + 0](_: forall z, z + 0 = z).
\end{lstlisting}
transforms the goal:
\begin{lstlisting}
  x + y + 0 = x + y + y + 0 + 0 + (x + y + 0)
\end{lstlisting}
into:
\begin{lstlisting}
  x + y + 0 = x + y + y + 0 + 0 + (x + y)
\end{lstlisting}
and generates a second subgoal:
\begin{lstlisting}
  forall z : nat, z + 0 = z
\end{lstlisting}
The second subgoal is generated by the use of an anonymous lemma in
the rewrite tactic. The effect of the tactic on the initial goal is to
rewrite this lemma at the second occurrence of the first matching
\L-x + y + 0- of the explicit rewrite redex \L-_ + y + 0-.


\subsubsection*{Occurrence selection and repetition}
Occurrence selection has priority over repetition switches. This means
the repetition of a rewrite tactic specified by a multiplier
will perform matching each time an elementary rewrite operation is
performed. Repeated rewrite tactics apply to every subgoal generated
by the previous tactic, including the previous instances of the
repetition. For example:
\begin{lstlisting}
  Goal forall x y z : nat,  x + 1 = x + y + 1.
  move=> x y z.
\end{lstlisting}
creates a goal \C{ x + 1 = x + y + 1}, which is turned into \C{z = z}
by the additional tactic:
\begin{lstlisting}
  rewrite 2!(_ : _ + 1 = z).
\end{lstlisting}
In fact, this last tactic generates \emph{three} subgoals,
respectively
\C{ x + y + 1 = z}, \C{ z = z} and \C{x + 1 = z}. Indeed, the second
rewrite operation specified with the \C{2!} multiplier applies to
the two subgoals generated by the first rewrite.

\subsubsection*{Multi-rule rewriting}
The \C{rewrite} tactic can be provided a \emph{tuple} of rewrite rules,
or more generally a tree of such rules, since this tuple can feature
arbitrary inner parentheses. We call \emph{multirule} such a
generalized rewrite rule. This feature is of special interest when it
is combined with  multiplier switches, which makes the \C{rewrite}
tactic iterates the rewrite operations prescribed by the rules on the
current goal. For instance, let us define two triples \C{multi1} and
\C{multi2} as:
\begin{lstlisting}
  Variables (a b c : nat).

  Hypothesis eqab : a = b.

  Hypothesis eqac : a = c.
\end{lstlisting}

Executing the tactic:
\begin{lstlisting}
  rewrite (eqab, eqac)
\end{lstlisting}
on the goal:
\begin{lstlisting}
  =========
  a = a
\end{lstlisting}
turns it into \C{b = b}, as rule \C{eqab} is the first to apply among
the ones gathered in the tuple passed to the \C{rewrite}
tactic. This multirule \C{(eqab, eqac)} is actually a \Coq{} term and we
can name it with a definition:
\begin{lstlisting}
Definition |*multi1*| := (eqab, eqac).
\end{lstlisting}
In this case, the tactic \C{rewrite multi1} is a synonym for
\C{(eqab, eqac)}. More precisely, a multirule rewrites
the first subterm to which one of the rules applies in a left-to-right
traversal of the goal, with the first rule from the multirule tree in
left-to-right order. Matching is performed according to the algorithm
described in Section~\ref{ssec:set}, but literal matches have
priority. For instance if we add a definition and a new multirule to
our context:

\begin{lstlisting}
  Definition |*d*| := a.

  Hypotheses eqd0 : d = 0.

  Definition |*multi2*| := (eqab, eqd0).
\end{lstlisting}
then executing the tactic:
\begin{lstlisting}
   rewrite multi2.
\end{lstlisting}
on the goal:
\begin{lstlisting}
  =========
  d = b
\end{lstlisting}
turns it into \C{0 = b}, as rule \C{eqd0} applies without unfolding
the definition of \C{d}.  For repeated rewrites the selection process
is repeated anew. For instance, if we define:

\begin{lstlisting}
   Hypothesis eq_adda_b : forall x, x + a = b.

   Hypothesis eq_adda_c : forall x, x + a = c.

   Hypothesis eqb0 : b = 0.

   Definition |*multi3*| := (eq_adda_b, eq_adda_c, eqb0).
\end{lstlisting}
then executing the tactic:
\begin{lstlisting}
  rewrite 2!multi3.
\end{lstlisting}
on the goal:
\begin{lstlisting}
  =========
  1 + a = 12 + a
\end{lstlisting}
turns it into \C{0 = 12 + a}: it uses \C{eq_adda_b} then \C{eqb0} on the
left-hand side only. Now executing the tactic \C{rewrite !multi3}
turns the same goal into \C{0 = 0}.

The grouping of rules inside a multirule does not affect the selection
strategy but can make it easier to include one rule set in another or
to (universally) quantify over the parameters of a subset of rules (as
there is special code that will omit unnecessary quantifiers for rules that
can be syntactically extracted). It is also possible to
reverse the direction of a rule subset, using a special dedicated syntax:
the tactic \C{rewrite (=~ multi1)} is equivalent to
\C{rewrite multi1_rev} with:
\begin{lstlisting}
  Hypothesis eqba : b = a.

  Hypothesis eqca : c = a.

  Definition |*multi1_rev*| := (eqba, eqca).
\end{lstlisting}
except that the constants \C{eqba, eqab, mult1_rev} have not been created.

Rewriting with multirules
is useful to implement simplification or transformation
procedures, to be applied on terms of small to medium size. For
instance the library \L{ssrnat} provides two implementations for
arithmetic operations on natural numbers: an elementary one and a tail
recursive version, less inefficient but also less convenient for
reasoning purposes. The library also provides one lemma per such
operation, stating that both versions return the same values when
applied to the same arguments:

\begin{lstlisting}
  Lemma |*addE*| : add =2 addn.
  Lemma |*doubleE*| : double =1 doublen.
  Lemma |*add_mulE*| n m s : add_mul n m s = addn (muln n m) s.
  Lemma |*mulE*| : mul =2 muln.
  Lemma |*mul_expE*| m n p : mul_exp m n p = muln (expn m n) p.
  Lemma |*expE*| : exp =2 expn.
  Lemma |*oddE*| : odd =1 oddn.
\end{lstlisting}

The operation on the left hand side of each lemma is the efficient
version, and the corresponding naive implementation is on the right
hand side. In order to reason conveniently on expressions involving
the efficient operations, we gather all these rules in the
definition \L+|*trecE*|+:
\begin{lstlisting}
 Definition |*trecE*| := (addE, (doubleE, oddE), (mulE, add_mulE, (expE, mul_expE))).
\end{lstlisting}
The tactic:
\begin{lstlisting}
  rewrite !trecE.
\end{lstlisting}
restores the naive versions of each operation in a goal involving the
efficient ones, e.g. for the purpose of a correctness proof.

\subsubsection*{Wildcards vs abstractions}
     The \C{rewrite} tactic supports r-items containing holes. For example
      in the tactic $(1)$:
\begin{lstlisting}
  rewrite (_ : _ * 0 = 0).
\end{lstlisting}
      the term \C{_ * 0 = 0} is interpreted as \C{forall n : nat, n * 0 = 0}.
      Anyway this tactic is \emph{not} equivalent to the tactic $(2)$:
\begin{lstlisting}
  rewrite (_ : forall x, x * 0 = 0).
\end{lstlisting}
      The tactic $(1)$ transforms the goal
      \L-(y * 0) + y * (z * 0) = 0- into \L+y * (z * 0) = 0+
      and generates a new subgoal to prove the statement \L+y * 0 = 0+,
      which is the \emph{instance} of the\\ \L+forall x, x * 0 = 0+
      rewrite rule that
      has been used to perform the rewriting. On the other hand, tactic
      $(2)$ performs the same rewriting on the current goal but generates a
      subgoal to prove \C{forall x, x * 0 = 0}.

\subsubsection*{When \ssr{} \C{rewrite} fails on standard \Coq{} licit rewrite}
In a few cases, the \ssr{} \C{rewrite} tactic fails
rewriting some redexes which standard \Coq{} successfully rewrites.
There are two main cases:

\begin{itemize}
\item \ssr{} never accepts to rewrite indeterminate patterns like:
\begin{lstlisting}
  Lemma |*foo*| : forall x : unit, x = tt.
\end{lstlisting}
\ssr{} will however accept the $\eta\zeta$ expansion of this rule:
\begin{lstlisting}
  Lemma |*fubar*| : forall x : unit, (let u := x in u) = tt.
\end{lstlisting}
\item In standard \Coq{}, suppose that we work in the following context:
\begin{lstlisting}
  Variable g : nat -> nat.
  Definition |*f*| := g.
\end{lstlisting}
then rewriting \C{H : forall x, f x = 0} in the goal
\C{g 3 + g 3 = g 6} succeeds
and transforms the goal into \C{0 + 0 = g 6}.

This rewriting is not possible in \ssr{} because there is no
occurrence of the head symbol \C{f} of the rewrite rule in the
goal. Rewriting with \C{H} first requires unfolding the occurrences of
\C{f} where the substitution is to be performed (here there is a single
such occurrence), using tactic \C{rewrite /f} (for a global
replacement of \C{f} by \C{g}) or \C{rewrite $\ \N{pattern}$/f}, for a
finer selection.
\end{itemize}

\subsubsection*{Existential metavariables and rewriting}
\label{ssec:rewcaveats}
The \C{rewrite} tactic will not instantiate existing existential
metavariables when matching a redex pattern.

If a rewrite rule generates a goal
with new existential metavariables, these will be generalized as for \C{apply}
(see page~\pageref{sssec:apply}) and corresponding new goals will be generated.
For example, consider the following script:

\begin{lstlisting}
  Lemma |*ex3*| (x : 'I_2) y (le_1 : y < 1) (E : val x = y) : Some x = insub y.
  rewrite insubT ?(leq_trans le_1)// => le_2.
\end{lstlisting}

Since \C{insubT} has the following type:

\begin{lstlisting}
  forall T P (sT : subType P) (x : T) (Px : P x), insub x = Some (Sub x Px)
\end{lstlisting}

and since the implicit argument corresponding to the \C{Px} abstraction is not
supplied by the user, the resulting goal should be \C{Some x = Some (Sub y
$\;\;?_{Px}$)}. Instead, \ssr{} \C{rewrite} tactic generates the two following
goals:
\begin{lstlisting}
  y < 2
  forall Hyp0 : y < 2, Some x = Some (Sub y Hyp0)
\end{lstlisting}
The script closes the former with \C{?(leq_trans le_1)//}, then it introduces
the new generalization naming it \C{le_2}.

\begin{lstlisting}
  x : 'I_2
  y : nat
  le_1 : y < 1
  E : val x = y
  le_2 : y < 2
  ============================
   Some x = Some (Sub y le_2)
\end{lstlisting}

As a temporary limitation, this behavior is available only if the rewriting
rule is stated using Leibniz equality (as opposed to setoid relations).
It will be extended to other rewriting relations in the future.

\subsection{Locking, unlocking }\label{ssec:lock}

As program proofs tend to generate large goals, it is important to be
able to control the partial evaluation performed by the simplification
operations that are performed by the tactics. These evaluations can
for example come from a \C{/=} simplification switch, or from rewrite steps
which may expand large terms while performing conversion. We definitely
want to avoid repeating large subterms of the goal in
the proof script. We do this by
``clamping down'' selected function symbols in the goal, which
prevents them from
being considered in simplification or rewriting steps. This clamping
is accomplished by using the occurrence switches (see section
\ref{sssec:occselect}) together with ``term tagging'' operations.

\ssr{} provides two levels of tagging.

The first one uses auxiliary definitions to introduce a provably equal
copy of any term \C{t}. However this copy is (on purpose)
\emph{not convertible} to \C{t} in the \Coq{} system\footnote{This is
  an implementation feature: there is not such obstruction in the
  metatheory}. The job is done by the following construction:

\begin{lstlisting}
  Lemma |*master_key*| : unit. Proof. exact tt. Qed.
  Definition |*locked*| A := let: tt := master_key in fun x : A => x.
  Lemma |*lock*| : forall A x, x = locked x :> A.
\end{lstlisting}
Note that the definition of \C{|*master_key*|} is explicitly opaque.
The equation \C{t = locked t} given by the \C{lock} lemma can be used
for selective rewriting, blocking on the fly the reduction in the
term \C{t}.
For example the script:
\begin{lstlisting}
  Require Import List.
  Variable A : Type.

  Fixpoint |*my_has*| (p : A -> bool)(l : list A){struct l} : bool:=
    match l with
      |nil => false
      |cons x l => p x || (my_has p l)
    end.

  Goal forall a x y l, a x = true -> my_has a ( x :: y :: l) = true.
  move=> a x y l Hax.
\end{lstlisting}
where \L{||} denotes the boolean disjunction, results in a goal
\L+my_has a ( x :: y :: l) = true+. The tactic:
\begin{lstlisting}
  rewrite {2}[cons]lock /= -lock.
\end{lstlisting}
turns it into \C{a x || my_has a (y :: l) = true}.
Let us now start by reducing the initial goal without blocking reduction.
The script:
\begin{lstlisting}
  Goal forall a x y l, a x = true -> my_has a ( x :: y :: l) = true.
  move=> a x y l Hax /=.
\end{lstlisting}
creates a goal \C{(a x) || (a y) || (my_has a l) = true}. Now the
tactic:
\begin{lstlisting}
  rewrite {1}[orb]lock orbC -lock.
\end{lstlisting}
where \L+|*orbC*|+ states the commutativity of \L+orb+, changes the
goal into\\ \L+(a x) || (my_has a l) || (a y) = true+: only the
arguments of the second disjunction where permuted.


It is sometimes desirable to globally prevent a definition from being
expanded by simplification; this is done by adding \L+locked+ in the
definition.

For instance, the function \L+|*fgraph_of_fun*|+ maps a function whose
domain and codomain are finite types to a concrete representation of
its (finite) graph. Whatever implementation of this transformation we
may use, we want it to be hidden to simplifications and tactics, to
avoid the collapse of the graph object:
\begin{lstlisting}
  Definition |*fgraph_of_fun*| :=
    locked
    (fun (d1 :finType) (d2 :eqType) (f : d1 -> d2) => Fgraph (size_maps f _)).
\end{lstlisting}

We provide a special tactic \C{unlock} for unfolding such definitions
while removing ``locks'', e.g., the tactic:
\begin{lstlisting}
  unlock $\N{occ-switch}$fgraph_of_fun.
\end{lstlisting}
replaces the occurrence(s) of \C{fgraph_of_fun} coded by the $\N{occ-switch}$
with \C{(Fgraph (size_maps _ _))} in the goal.

We found that it was usually preferable to prevent the expansion of
some functions by the partial evaluation switch ``/='', unless
this allowed the evaluation of a condition. This is possible thanks to
an other mechanism of term tagging, resting on the following
\emph{Notation}:
\begin{lstlisting}
  Notation "'nosimpl' t" := (let: tt := tt in t).
\end{lstlisting}

The term \C{(nosimpl t)} simplifies to t \emph{except} in a
definition. More precisely,
given:
\begin{lstlisting}
  Definition |*foo*| := (nosimpl bar).
\end{lstlisting}
the term \C{foo (or (foo t'))} will \emph{not} be expanded by the
\emph{simpl} tactic unless it is in a forcing context (e.g., in
\C{match foo t' with $\dots$ end}, \C{foo t'} will be reduced if this allows
\C{match} to be reduced). Note that \C{nosimpl bar} is simply notation
for a term that reduces to \C{bar}; hence \C{unfold foo} will replace
    \C{foo} by \C{bar}, and \C{fold foo} will replace \C{bar} by
    \C{foo}.

\emph{Warning} The \C{nosimpl} trick only works if no reduction is
apparent in \C{t}; in particular, the declaration:
\begin{lstlisting}
  Definition |*foo*| x := nosimpl (bar x).
\end{lstlisting}
will usually not work. Anyway, the common practice is to tag only the
function, and to use the following definition, which blocks the
reduction as expected:
\begin{lstlisting}
  Definition |*foo*| x := nosimpl bar x.
\end{lstlisting}


A standard example making this technique shine is the case of
arithmetic operations. We define for instance:
\begin{lstlisting}
  Definition |*addn*| := nosimpl plus.
\end{lstlisting}
The operation \C{addn} behaves exactly like plus, except that
\L+(addn (S n) m)+ will not
simplify spontaneously to \L+(S (addn n m))+ (the two terms, however, are
inter-convertible). In addition, the unfolding step:
\begin{lstlisting}
rewrite /addn
\end{lstlisting}
will replace \C{addn} directly with \C{plus}, so the \C{nosimpl} form
is essentially invisible.

\subsection{Congruence}\label{ssec:congr}

Because of the way matching interferes with type families parameters,
the tactic:
\begin{lstlisting}
  apply: my_congr_property.
\end{lstlisting}
will generally fail to perform congruence simplification, even on
rather simple cases. We therefore provide a
more robust alternative in which the function is supplied:
$$\C{congr}\ [\N{int}]\ \N{term}$$

This tactic:
\begin{itemize}
\item checks that the goal is a Leibniz equality
\item matches both sides of this equality with ``$\N{term}$ applied to
  some arguments'', inferring the right number of arguments from the goal
  and the type of $\N{term}$. This may
  expand some definitions or fixpoints.
\item generates the subgoals corresponding to pairwise equalities of
  the arguments present in the goal.
\end{itemize}

The goal can be a non dependent product \C{P -> Q}.
In that case, the system asserts the equation \C{P = Q}, uses it to solve
the goal, and calls the \C{congr} tactic on the remaining goal
\C{P = Q}. This can be useful for instance to perform a transitivity
step, like in the following situation:
\begin{lstlisting}
  x, y, z : nat
  ===============
  x = y -> x = z
\end{lstlisting}
the tactic \C{congr (_ = _)} turns this goal into:

\begin{lstlisting}
  x, y, z : nat
  ===============
  y = z
\end{lstlisting}
which can also be obtained starting from:
\begin{lstlisting}
  x, y, z : nat
  h : x = y
  ===============
  x = z
\end{lstlisting}
and using the tactic \C{congr (_ = _): h}.

The optional $\N{int}$ forces the number of arguments for which the
tactic should generate equality proof obligations.

This tactic supports equalities between applications with dependent
arguments. Anyway as in standard
\Coq{}, dependent arguments should have exactly the same
parameters on both sides, and these parameters should appear as first
arguments.

The following script:
\begin{lstlisting}
  Definition f n := match n with 0 => plus | S _ => mult end.
  Definition g (n m : nat) := plus.

  Goal forall x y, f 0 x y = g 1 1 x y.
  by move=> x y; congr plus.
  Qed.
\end{lstlisting}
shows that the \L+congr+ tactic matches \L+plus+ with \L+f 0+ on the
left hand side and \L+g 1 1+ on the right hand side, and solves the goal.

The script:
\begin{lstlisting}
  Goal forall n m, m <= n -> S m + (S n - S m) = S n.
  move=> n m Hnm; congr S; rewrite -/plus.
\end{lstlisting}
generates the subgoal \C{m + (S n - S m) = n}. The tactic
\C{rewrite -/plus} folds back the expansion of \C{plus} which was
necessary for matching both sides of the equality with an application
of \C{S}.

Like most \ssr{} arguments, $\N{term}$ can contain wildcards.
The script:
\begin{lstlisting}
  Goal forall x y, x + (y * (y + x - x)) = x * 1 + (y + 0) * y.
  move=> x y; congr ( _ + (_ * _)).
\end{lstlisting}
generates three subgoals, respectively \C{x = x * 1}, \C{y = y + 0}
and \C{ y + x - x = y}.
