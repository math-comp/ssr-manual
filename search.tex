\section{\ssr{} searching tool}

\ssr{} proposes an extension of the \C{Search} command. Its syntax is:


$$\C{Search}\ [\N{pattern}]\ [\ [\C{\-}][\ \N{string}[\C{\%}\N{key}]\
|\ \N{pattern}]\ ]^*\
[\C{in}\ [\ [\C{\-}]\N{name}\ ]^+]\C{.}$$

% \begin{lstlisting}
%   Search [[\~]$\N{string}$]$^*$ [$\N{pattern}$] [[$\N{pattern}_1 \dots $ $\N{pattern}_n$]] $[[$inside$|$outside$]$ $M_1 \dots M_n$].
% \end{lstlisting}

% This tactic returns the list of defined constants matching the
%  given criteria:
% \begin{itemize}
% \item \L+[[-]$\N{string}$]$^*$+ is an open sequence of strings, which sould
%   all appear in the name of the returned constants. The optional \L+-+
%   prefixes strings that are required  \emph{not} to appear.
% % \item $\N{pattern}$ should be a subterm of the
% %   \emph{conclusion} of the lemmas found by the command. If a lemma features
% %   an occurrence
% %   of this pattern only in one or several of its assumptions, it will not be
% %   selected by the searching tool.
% \item
% \L=[$\N{pattern}^+$]=
% is a list of \ssr{} terms, which may
% include types, that are required to appear in the returned constants.
% Terms with holes should be surrounded by parentheses.
% \item $\C{in}\ [[\C{\-}]M]^+$ limits the search to the signature
%   of open modules given in the list, but the ones preceeded by the
%   $\C{\-}$ flag. The
%   command:
% \begin{lstlisting}
%   Search in M.
% \end{lstlisting}
% is hence a way of obtaining the complete signature of the module \L{M}.
% \end{itemize}
where $\N{name}$ is the name of an open module.
This command search returns the list of lemmas:
\begin{itemize}
\item whose \emph{conclusion} contains a subterm matching the optional
  first $\N{pattern}$. A $\C{-}$ reverses the test, producing the list
  of lemmas whose conclusion does not contain any subterm matching
  the pattern;
\item whose name contains the given string. A $\C{-}$ prefix reverses
  the test, producing the list of lemmas whose name does not contain the
  string. A string that contains symbols or
is followed by a scope $\N{key}$, is interpreted as the constant whose
notation involves that string (e.g., \L=+= for \L+addn+), if this is
unambiguous; otherwise the diagnostic includes the output of the
\C{Locate} vernacular command.

\item whose statement, including assumptions and types, contains a
  subterm matching the next patterns. If a pattern is prefixed by
  $\C{-}$, the test is reversed;
\item contained in the given list of modules, except the ones in the
  modules prefixed by a $\C{-}$.
\end{itemize}

Note that:
\begin{itemize}
\item As for regular terms, patterns can feature scope
  indications. For instance, the command:
\begin{lstlisting}
  Search _ (_ + _)%N.
\end{lstlisting}
lists all the lemmas whose statement (conclusion or hypotheses)
involve an application of the binary operation denoted by the infix
\C{+} symbol in the \C{N} scope (which is \ssr{} scope for natural numbers).
\item Patterns with holes should be surrounded by parentheses.
\item Search always volunteers the expansion of the notation, avoiding the
  need to execute Locate independently. Moreover, a string fragment
  looks for any notation that contains fragment as
  a substring. If the \L+ssrbool+ library is imported, the command:
\begin{lstlisting}
  Search "~~".
\end{lstlisting}
answers :
\begin{lstlisting}
"~~" is part of notation (~~ _)
In bool_scope, (~~ b) denotes negb b
negbT  forall b : bool, b = false -> ~~ b
contra  forall c b : bool, (c -> b) -> ~~ b -> ~~ c
introN  forall (P : Prop) (b : bool), reflect P b -> ~ P -> ~~ b
\end{lstlisting}
 \item A diagnostic is issued if there are different matching notations;
  it is an error if all matches are partial.
\item Similarly, a diagnostic warns about multiple interpretations, and
  signals an error if there is no default one.
\item The command  \C{Search in M.}
is a way of obtaining the complete signature of the module \L{M}.
\item Strings and pattern indications can be interleaved, but the
  first indication has a special status if it is a pattern, and only
  filters the conclusion of lemmas:
\begin{itemize}
  \item The command :
    \begin{lstlisting}
      Search (_ =1 _) "bij".
    \end{lstlisting}
lists all the lemmas whose conclusion features a '$\C{=1}$' and whose
name contains the string \verb+bij+.
\item The command :
    \begin{lstlisting}
      Search  "bij" (_ =1 _).
    \end{lstlisting}
lists all the lemmas whose statement, including hypotheses, features a
'$\C{=1}$' and whose name contains the string \verb+bij+.

\end{itemize}

\end{itemize}
