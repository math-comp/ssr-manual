\section{Distribution}

\subsection{Files}\label{sec:files}
The Ssreflect package distributes the implementation of the
small-scale reflection extension. It comprises
two Caml extension modules ({\tt ssreflect.ml4} and {\tt
  ssrmatching.ml4}), which provides the tactic language, as well as several
\Coq{} vernacular files:
\begin{itemize}
\item {\tt ssreflect}, {\tt ssrmatching}: technical results for \ssr{} tactics
\item {\tt ssrfun}: functions, functional equality, bijectivity,
  injectivity, surjectivity,...
\item {\tt ssrbool}: extended toolkit for reflection
\item {\tt eqtype}: structure for type with decidable intentional equality
\item {\tt choice}: structures for types with a choice operator and for
  countable types
\item {\tt ssrnat}: natural numbers, arithmetic
\item {\tt seq}: lists
\item {\tt fintype}: structure for finite types
          (types with finitely many elements)
\end{itemize}
% Note that some files from the \ssr{}~1.2 distribution have been renamed
% (trailing {\tt s} are dropped, and {\tt connect} becomes {\tt
%   fingraph}).
More \Coq{} vernacular files are available from the
Mathematical Components and Odd Order Theorem packages.


\subsection{Compatibility issues}\label{sec:compat}
Every effort has been made to make the small-scale reflection
extensions upward compatible with the basic \Coq{}, but a few
discrepancies were unavoidable:
\begin{itemize}
\item New keywords (\C{is}) might clash with variable, constant,
tactic or tactical names, or with quasi-keywords in tactic or
vernacular notations.
\item New tactic(al)s names (\L+last+, \L+done+, \L+have+,
  \L+suffices+, \L+suff+,
  \L+without loss+, \L+wlog+, \L+congr+, \L+unlock+) might clash
  with user tactic names.
\item Identifiers with both leading and trailing \L+_+, such as \L+_x_+,
are reserved by \ssr{} and cannot appear in scripts.
\item The extensions to the \L+rewrite+ tactic are partly
incompatible with those now available in current versions of \Coq{};
in particular:
\L+rewrite .. in (type of k)+ or \\ \L+rewrite .. in *+ or any other
variant of \L+rewrite+ will not work, and the \ssr{} syntax and semantics for occurrence selection and
rule chaining is different.
% Moreover \ssr{} looks for redexes more aggressively than $\mathcal{L}$-tac.
Use an explicit rewrite direction (\L+rewrite <-+ ... or \L+rewrite ->+ ...)
to access the \Coq{} \L+rewrite+ tactic.
\item New symbols (\L+//, /=, //=+) might clash with adjacent symbols
(e.g., '\L+//+') instead of '$/$''$/$'). This can be avoided by
inserting white spaces.
\item New constant and theorem names might clash with the user
theory. This can be avoided by not importing all of \ssr{}:
\begin{lstlisting}
  Require ssreflect.
  Import ssreflect.SsrSyntax.
\end{lstlisting}
Note that \ssr{} extended rewrite syntax and reserved identifiers are
enabled only if the \C{ssreflect} module has been required and if
\C{SsrSyntax} has been imported. Thus a file that requires (without importing)
 \C{ssreflect} and imports \C{SsrSyntax}, can be
required and imported without automatically enabling \ssr{}'s
extended rewrite syntax and reserved identifiers.
\item Some user notations (in particular, defining an infix ';') might
interfere with "open term" syntax of tactics such as \L+have+, \L+set+
and \L+pose+.
\item The generalization of \L+if+ statements to non-Boolean
conditions is turned off by \ssr{}, because it is mostly subsumed by
\L+Coercion+ to \L+bool+ of the \L+sum+XXX types (declared in
\L+ssrfun.v+) and the \L+if ... is+ construct (see
\ref{ssec:patcond}). To use the generalized form, turn off the \ssr{}
Boolean \L+if+ notation using the command:
\begin{lstlisting}
  Close Scope boolean_if_scope.
\end{lstlisting}
\item The following two options can be unset to disable the
      incompatible \L+rewrite+ syntax and allow
      reserved identifiers to appear in scripts.
\begin{lstlisting}
  Unset SsrRewrite.
  Unset SsrIdents.
\end{lstlisting}
\end{itemize}
