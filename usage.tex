\section{Usage}

\subsection{Getting started}\label{sec:files}
% The present manual describes the behavior of
% the \ssr{} extension after having loaded a minimal set of libraries:
% {\tt ssreflect.v}, {\tt ssrfun.v} and {\tt ssrbool.v}. For instance,
% with a standard installation of the \ssr{} package, it is sufficient
% to execute the following command:

% \begin{lstlisting}
%   From mathcomp Require Import ssreflect ssrfun ssrbool.
% \end{lstlisting}

\todo{Here should come the prelude incantation which provides the
  behaviour described in the present manual.}

\subsection{Compatibility issues}\label{sec:compat}
Every effort has been made to make the small-scale reflection
extensions upward compatible with the rest of \Coq{}, but a few
discrepancies were unavoidable:
\begin{itemize}
\item New keywords (\C{is}) might clash with variable, constant,
tactic or tactical names, or with quasi-keywords in tactic or
vernacular notations.
\item New tactic(al)s names (\L+last+, \L+done+, \L+have+,
  \L+suffices+, \L+suff+,
  \L+without loss+, \L+wlog+, \L+congr+, \L+unlock+) might clash
  with user tactic names.
\item Identifiers with both leading and trailing \L+_+, such as \L+_x_+,
are reserved by \ssr{} and cannot appear in scripts.
\item The extensions to the \L+rewrite+ tactic are partly
incompatible with those available in current versions of \Coq{};
in particular:
\L+rewrite .. in (type of k)+ or \\ \L+rewrite .. in *+ or any other
variant of \L+rewrite+ will not work, and the \ssr{} syntax and semantics for occurrence selection and
rule chaining is different.
% Moreover \ssr{} looks for redexes more aggressively than \Ltac{}.
Use an explicit rewrite direction (\L+rewrite <-+ $\dots$ or \L+rewrite ->+ $\dots$)
to access the \Coq{} \L+rewrite+ tactic.
\item New symbols (\L+//, /=, //=+) might clash with adjacent existing
  symbols (e.g., '\L+//+') instead of '\L+/+''\L+/+'). This can be avoided
  by inserting white spaces.
\item New constant and theorem names might clash with the user
theory. This can be avoided by not importing all of \ssr{}:
\begin{lstlisting}
  From mathcomp Require ssreflect.
  Import ssreflect.SsrSyntax.
\end{lstlisting}
Note that \ssr{}'s extended rewrite syntax and reserved identifiers are
enabled only if the \C{ssreflect} module has been required and if
\C{SsrSyntax} has been imported. Thus a file that requires (without importing)
 \C{ssreflect} and imports \C{SsrSyntax}, can be
required and imported without automatically enabling \ssr{}'s
extended rewrite syntax and reserved identifiers.
\item Some user notations (in particular, defining an infix ';') might
interfere with the "open term", parenthesis free, syntax of tactics
such as \L+have+, \L+set+ and \L+pose+.
\item The generalization of \L+if+ statements to non-Boolean
conditions is turned off by \ssr{}, because it is mostly subsumed by
\L+Coercion+ to \L+bool+ of the \L+sum+XXX types (declared in
\L+ssrfun.v+)
\todo{Is this relevant to the partial merge?}
 and the \C{if $\ \N{term}\ $ is $\ \N{pattern}\ $
  then $\ \N{term}\ $ else $\ \N{term}\ $} construct (see
\ref{ssec:patcond}). To use the generalized form, turn off the \ssr{}
Boolean \L+if+ notation using the command:
\begin{lstlisting}
  Close Scope boolean_if_scope.
\end{lstlisting}
\item The following two options can be unset to disable the
      incompatible \L+rewrite+ syntax and allow
      reserved identifiers to appear in scripts.
\begin{lstlisting}
  Unset SsrRewrite.
  Unset SsrIdents.
\end{lstlisting}
\todo{Say something about bullets?}
\end{itemize}
